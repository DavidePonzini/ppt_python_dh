\makesectionframe{Cicli}

\begin{contentframe}
    \frametitle{Cicli}

    \begin{itemize}
        \item Permettono di ripetere le stesse azioni più volte

        \bigskip
        \item Due costrutti simili ed intercambiabili
        \begin{itemize}
            \item Usiamo quello più comodo per quello che dobbiamo fare
        \end{itemize}
        
        \bigskip
        \item Costrutto \texttt{while} per iterare fino a che una certa condizione non diventa falsa
        \item Costrutto \texttt{for} per iterare su una serie di valori o un certo numero di volte
    \end{itemize}
\end{contentframe}

\begin{contentframe}
    \frametitle{Ciclo while}

    \begin{columns}
        \col{.5}
        \begin{itemize}
            \item Eseguo tutte le istruzioni all'interno del ciclo fino a che la condizione è vera
            \item La condizione viene controllata prima di iniziare ogni iterazione

            \bigskip
            \item Consigliato quando ho una condizione booleana facilmente comprensibile
        \end{itemize}
        
        \col{.5}
        \centering
        \image{while.jpg}
    \end{columns}
\end{contentframe}

\begin{contentframe}
    \frametitle{Ciclo for}

    \begin{columns}
        \col{.5}
        \begin{itemize}
            \item Usato per iterare su una sequenza di valori
            \item La variabile \texttt{value} (possiamo chiamarla come vogliamo) assume ad ogni iterazione un valore diverso

            \bigskip
            \item Consigliato quando ho una sequenza di elementi
        \end{itemize}
        
        \col{.5}
        \centering
        \image{for.jpg}
    \end{columns}
\end{contentframe}

\begin{contentframe}
    \frametitle{While vs for (1)}

    \begin{itemize}
        \item I seguenti codici sono identici, ma uno è più facile da comprendere
    \end{itemize}

    \begin{columns}
        \col{.5}
        \centering
        \image{while_example.jpg}
        
        \col{.5}
        \centering
        \image{for_example.jpg}
    \end{columns}
\end{contentframe}

\begin{contentframe}
    \frametitle{While vs for (2)}

    \begin{itemize}
        \item I seguenti codici sono identici, ma uno è più facile da comprendere
    \end{itemize}

    \begin{columns}
        \col{.5}
        \centering
        \image{while_example2.jpg}
        
        \col{.5}
        \centering
        \image{for_example2.jpg}
    \end{columns}
\end{contentframe}

\begin{contentframe}
    \frametitle{Break \& continue}

    \begin{itemize}
        \item Parole chiave speciali che si possono usare nei cicli

        \bigskip
        \item \boxed{\texttt{break}} interrompe immediatamente l'esecuzione del ciclo
        \begin{itemize}
            \item Esegue la prima istruzione fuori dal ciclo
            \item Utile se una volta incontrato un certo elemento non vogliamo più far nulla
        \end{itemize}

        \bigskip
        \item \boxed{\texttt{continue}} fa saltare il giro corrente
        \begin{itemize}
            \item Esegue la prima azione del ciclo col valore successivo
            \item Utile se per un certo elemento non vogliamo eseguire alcuna azione ma per quelli dopo sì
        \end{itemize}
    \end{itemize}
\end{contentframe}

\begin{exampleframe}
    \frametitle{Break}
    \framesubtitle{Esempio}

    \image*{break.jpg}
\end{exampleframe}

\begin{exampleframe}
    \frametitle{Continue}
    \framesubtitle{Esempio}

    \image*{continue.jpg}
\end{exampleframe}

\begin{exerciseframe}
    \frametitle{Esercizi (1)}
    \framesubtitle{Cicli}

    \begin{itemize}
        \item Stampare tutti i numeri pari da 0 (compreso) a 500 (escluso)\pause, usando ciascuno dei seguenti metodi
        \begin{itemize}
            \item Ciclo \texttt{while}, con un contatore che parte da 1 ed incrementa di 2 ogni volta
            \pause
            \item Ciclo \texttt{while}, con un contatore che incrementa di 1 ed un \texttt{if} che stampa solo i numeri pari
            \pause
            \item Ciclo \texttt{while}, con un contatore che incrementa di 1 e un \texttt{if} che salta i numeri dispari
            \pause
            \item Ciclo \texttt{for}, con l'aiuto della funzione \texttt{range(0, 500)} ed un \texttt{if}
            \pause
            \item Ciclo \texttt{for}, con l'aiuto della funzione \texttt{range(500)} ed un \texttt{if}
            \pause
            \item Ciclo \texttt{for}, con l'aiuto della funzione \texttt{range(0, 500, 2)}
        \end{itemize}
    \end{itemize}
\end{exerciseframe}

\begin{exerciseframe}
    \frametitle{Esercizi (2)}
    \framesubtitle{Cicli}

    \begin{itemize}
        \bigskip
        \item Chiedere un numero $x$ a scelta dell'utente e stampare la tabellina di quel numero (da $x*0$ a $x*10$)

        \pause
        \bigskip
        \item Chiedere all'utente due numeri ($x$ e $n$) e stampare tutti i multipli di $x$ minori o uguali ad $n$
    \end{itemize}
\end{exerciseframe}


\makesectionframe{Strutture dati}

\begin{contentframe}
    \frametitle{Strutture dati}

    \begin{itemize}
        \item Permettono di memorizzare dati in formati specifici

        \bigskip
        \item Le principali sono:
        \begin{itemize}
            \item Liste -- sequenza di elementi
            \item Set -- sequenza di elementi senza ripetizioni
            \item Dizionari -- associazione chiave-valore
        \end{itemize}
    \end{itemize}
\end{contentframe}

\begin{contentframe}
    \frametitle{Liste}

    \begin{itemize}
        \item Sequenza di elementi ordinata

        \bigskip
        \item Posso accedere ad un elemento specifico sapendone la sua posizione
        \item Attenzione: le posizioni partono da 0
        \begin{itemize}
            \item Il primo elemento si trova in pos. 0
            \item Il secondo elemento si trova in pos. 1
            \item ...
            \item Il decimo elemento si trova in pos. 9
        \end{itemize}
    \end{itemize}
\end{contentframe}

\begin{contentframe}
    \frametitle{Liste}
    \framesubtitle{Operazioni}

    \only<1>{\image[1][.75]{list_op1.jpg}}
    \only<2>{\image[1][.75]{list_op2.jpg}}
    \only<3>{\image[1][.75]{list_op3.jpg}}
\end{contentframe}

\begin{contentframe}
    \frametitle{Liste}
    \framesubtitle{Splicing}

    \begin{itemize}
        \item Permette di ottenere un sottoinsieme della lista
    
        \bigskip
        \item Si usa l'operatore \texttt{[...]} con due o tre argomenti, separati da ``\texttt{:}''
        \begin{itemize}
            \item \texttt{lista[start:end]}
            \item \texttt{lista[start:end:step]}
        \end{itemize}
        
        \item Seleziona tutti gli elementi da \texttt{start} (compreso) a \texttt{end} (escluso)
        \begin{itemize}
            \item Se \texttt{step} è presente, si salta di \texttt{step} valori alla volta
        \end{itemize}
    \end{itemize}
\end{contentframe}

\begin{contentframe}
    \frametitle{Liste}
    \framesubtitle{Splicing}

    \only<1>{\image[1][.75]{list_op4.jpg}}
    \only<2>{\image[1][.75]{list_op5.jpg}}
\end{contentframe}

\begin{exerciseframe}
    \frametitle{Esercizi (1)}
    \framesubtitle{Liste}

    \begin{enumerate}
        \item Chiedere all'utente di inserire 10 numeri in input. Al termine, stampare i numeri nello stesso ordine in cui sono stati inseriti

        \pause
        \bigskip
        \item Chiedere all'utente di inserire 10 numeri in input. Al termine, stampare i numeri in ordine inverso rispetto a come sono stati inseriti

        \pause
        \bigskip
        \item Chiedere all'utente di inserire 5 numeri in input. Al termine, chiedere un nuovo numero e dire se era già stato inserito
    \end{enumerate}
\end{exerciseframe}

\begin{exerciseframe}
    \frametitle{Esercizi (2)}
    \framesubtitle{Liste}

    \begin{columns}
        \col{.5}
        \begin{enumerate}
            \setcounter{enumi}{3}
            \item Chiedere all'utente di inserire dei numeri in input, fino a che non viene inserito 0. Infine, stampare:
            \begin{itemize}
                \item quanti numeri sono stati inseriti
                \only<2->{
                    \item il valore minimo
                    \item il valore massimo
                    \item la media
                }
                \only<3->{
                    \item quanti numeri sono negativi
                }
                \only<4->{
                    \item quanti numeri sono negativi nella seconda metà della sequenza \textit{(suggerimento: utilizzare la divisione tra interi)}
                }
            \end{itemize}
        \end{enumerate}
        
        \col{.5}
        \centering
        \image[1][.7]{list_exercise.jpg}
    \end{columns}
    
\end{exerciseframe}

\begin{contentframe}
    \frametitle{Set}

    \begin{itemize}
        \item Sequenza di elementi non ordinata e senza ripetizioni

        \bigskip
        \item Sono disponibili le stesse operazioni per le liste (con piccole eccezioni)
        \item Non posso usare l'operatore ``\texttt{[...]}''
    \end{itemize}
\end{contentframe}

\begin{contentframe}
    \frametitle{Set}
    \framesubtitle{Operazioni}

    \only<1>{\image[1][.75]{set_op1.jpg}}
    \only<2>{\image[1][.75]{set_op2.jpg}}
 \end{contentframe}

\begin{exerciseframe}
    \frametitle{Esercizi}
    \framesubtitle{Set}

    \begin{enumerate}
        \item Chiedere all'utente di inserire dei nomi, fino a che non scrive ``stop''. Infine, stampare tutti i nomi inseriti. Nel caso un nome sia inserito più volte, vogliamo stamparlo una volta sola

        \bigskip
        \item Chiedere all'utente di inserire dei nomi, fino a che non scrive ``stop''. Infine, stampare il nome più lungo
    \end{enumerate}
\end{exerciseframe}

\begin{contentframe}
    \frametitle{Dizionari}

    \begin{itemize}
        \item Associano ad ogni chiave un valore
        \begin{itemize}
            \item \textit{Esempio: associo ad ogni studente il suo voto}
        \end{itemize}

        \bigskip
        \item Si può immaginare come l'unione tra set e liste
        \begin{itemize}
            \item Le chiavi si comportano come un set
            \begin{itemize}
                \item \textit{Non posso avere due studenti con lo stesso nome}
            \end{itemize}
            \item I valori si comportano come una lista
            \begin{itemize}
                \item \textit{Due studenti diversi possono avere lo stesso voto}
            \end{itemize}
        \end{itemize}
    \end{itemize}
\end{contentframe}

\begin{contentframe}
    \frametitle{Dizionari}

    \begin{itemize}
        \item Ad ogni chiave può essere associato un solo valore
        \item Se tento di assegnare un nuovo valore quando è già presente, sovrascriverò quello vecchio

        \bigskip
        \item I valori possono essere recuperati solo conoscendo la chiave a cui sono stati assegnati

        \bigskip
        \item Non possono esistere due chiavi identiche (visto che si usano per ricercare i valori)
        \item Due chiavi diverse possono però avere lo stesso valore 

        \bigskip
        \item Se tento di assegnare un valore ad una chiave che non esiste, in automatico creo anche la chiave corrispondente 
    \end{itemize}
\end{contentframe}

\begin{contentframe}
    \frametitle{Dizionari}
    \framesubtitle{Operazioni}

    \only<1>{\image[1][.75]{dict_op1.jpg}}
    \only<2>{\image[1][.75]{dict_op2.jpg}}
    \only<3>{\image[1][.75]{dict_op3.jpg}}
    \only<4>{\image[1][.75]{dict_op4.jpg}}
\end{contentframe}

\begin{exampleframe}
    \frametitle{Note}

    \begin{itemize}
        \item Nelle liste, set e mappe possiamo usare qualsiasi tipo di dato

        \bigskip
        \item In una stessa struttura dati posso usare dati di tipi diversi
            \image*[.7]{dict_note.jpg}
            
    \end{itemize}
\end{exampleframe}

\begin{exerciseframe}
    \frametitle{Esercizi (0)}
    \framesubtitle{Dizionari}

    Creare il seguente dizionario:
    
    \bigskip
    \image[.8]{dict_exercise.jpg}
\end{exerciseframe}

\begin{exerciseframe}
    \frametitle{Esercizi (1)}
    \framesubtitle{Dizionari}

    \begin{columns}
        \col{.6}
        \begin{enumerate}
            \item Stampare i voti di Paolo
            \item Stampare la media di Paolo
            \item Stampare le medie di tutti e tre gli studenti, usando un ciclo \texttt{for}
            \item Stampare i singoli voti di tutti e tre gli studenti, usando due cicli \texttt{for}, in modo simile all'immagine
        \end{enumerate}
        
        \col{.4}
        \centering
        \image[1][.5]{dict_exercise2.jpg}
    \end{columns}
\end{exerciseframe}

\begin{exerciseframe}
    \frametitle{Esercizi (2)}
    \framesubtitle{Dizionari}

    \begin{enumerate}
        \setcounter{enumi}{4}
        \item Chiedere all'utente di inserire delle parole, fino a che non scrive ``stop''. Infine, stampare tutte le parole inserite, assieme al numero di volte che sono state inserite
    \end{enumerate}
\end{exerciseframe}
