\begin{contentframe}
    \frametitle{Condizioni}

    \begin{columns}
        \col{.5}
        \begin{itemize}
            \item Permettono di eseguire delle azioni solo in certi casi
    
            \bigskip
            \item Sintassi:
            \begin{itemize}
                \item Dopo \texttt{if} c'è uno spazio
                \item La condizione ritorna un booleano (\texttt{True}/\texttt{False})
                \item Dopo la condizione ci sono i due punti (``\texttt{:}'')
                \item Le istruzioni eseguite all’interno della condizione sono indentate (con un tab o 4 spazi)
            \end{itemize}
        \end{itemize}
        
        \col{.5}
        \centering
        \image{if.jpg}
    \end{columns}
\end{contentframe}

\begin{contentframe}
    \frametitle{Condizioni}

    \begin{itemize}
        \item Se ho più condizioni, posso anche usare \texttt{else} o \texttt{elif} (\textit{else if})
    \end{itemize}

    \bigskip
    \centering
    \only<1>{
        \image{if_else.jpg}
    }
    \only<2>{
        \image[.5]{if_elif_bad.jpg}
    }
    \only<3>{
        \image[.5]{if_elif_good.jpg}
        
        \bigskip
        \texttt{elif} permette di mantenere il codice più facilmente leggibile
    }
    
\end{contentframe}

\begin{exerciseframe}
    \frametitle{Condizioni}

    \begin{itemize}
        \item Chiedere all'utente in quale città abita
        \item Se la risposta è ``Genova'', rispondere con ``Anche io!''
        \item Altrimenti, rispondere con ``Io invece abito a Genova''

        \pause
        \bigskip
        \item Accettare come risposte corrette ``Genova'', ``genova'' e ``GENOVA''
        
        \pause
        \bigskip
        \item Opzionale: accettare tutte le possibili variazioni di case
    \end{itemize}
\end{exerciseframe}

\begin{contentframe}
    \frametitle{Cicli}

    \begin{itemize}
        \item Permettono di ripetere le stesse azioni più volte

        \bigskip
        \item Due costrutti simili ed intercambiabili
        \begin{itemize}
            \item Usiamo quello più comodo per quello che dobbiamo fare
        \end{itemize}
        
        \bigskip
        \item Costrutto \texttt{while} per iterare fino a che una certa condizione non diventa falsa
        \item Costrutto \texttt{for} per iterare su una serie di valori o un certo numero di volte
    \end{itemize}
\end{contentframe}

\begin{contentframe}
    \frametitle{Cicli}
    \frametitle{While}

    \begin{columns}
        \col{.5}
        \begin{itemize}
            \item Eseguo tutte le istruzioni all'interno del ciclo fino a che la condizione è vera
            \item La condizione viene controllata prima di iniziare ogni iterazione

            \bigskip
            \item Consigliato quando ho una condizione booleana facilmente comprensibile
        \end{itemize}
        
        \col{.5}
        \centering
        \image{while.jpg}
    \end{columns}
\end{contentframe}

\begin{contentframe}
    \frametitle{Cicli}
    \frametitle{For}

    \begin{columns}
        \col{.5}
        \begin{itemize}
            \item Usato per iterare su una sequenza di valori
            \item La variabile \texttt{x} (possiamo chiamarla come vogliamo) assume ad ogni iterazione un valore diverso

            \bigskip
            \item Consigliato quando ho una sequenza di elementi
        \end{itemize}
        
        \col{.5}
        \centering
        \image{for.jpg}
    \end{columns}
\end{contentframe}

\begin{contentframe}
    \frametitle{Cicli}
    \frametitle{While vs for (1)}

    \begin{itemize}
        \item I seguenti codici sono identici, ma uno è più facile da comprendere
    \end{itemize}

    \begin{columns}
        \col{.5}
        \centering
        \image{while_example.jpg}
        
        \col{.5}
        \centering
        \image{for_example.jpg}
    \end{columns}
\end{contentframe}

\begin{contentframe}
    \frametitle{Cicli}
    \frametitle{While vs for (2)}

    \begin{itemize}
        \item I seguenti codici sono identici, ma uno è più facile da comprendere
    \end{itemize}

    \begin{columns}
        \col{.5}
        \centering
        \image{while_example2.jpg}
        
        \col{.5}
        \centering
        \image{for_example2.jpg}
    \end{columns}
\end{contentframe}

\begin{exerciseframe}
    \frametitle{Cicli}

    \begin{itemize}
        \item Chiedere un numero $x$ a scelta dell'utente e stampare la tabellina di quel numero (da $x*0$ a $x*10$)

        \pause
        \bigskip
        \item Chiedere all'utente due numeri ($x$ e $n$) e stampare tutti i multipli di $x$ minori o uguali ad $n$
    \end{itemize}
\end{exerciseframe}