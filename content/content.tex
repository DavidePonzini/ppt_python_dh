\makesectionframe{Presentazione corso}

\begin{contentframe}
    \frametitle{Obiettivi del corso}

    \begin{itemize}
        \item Imparare i concetti base ed intermedi di Python
        
        \bigskip
        \item Imparare ad apprendere autonomamente nuovi concetti

        \bigskip
        \item Imparare le basi del pensiero computazionale
    \end{itemize}
\end{contentframe}

\begin{contentframe}
    \frametitle{Contenuti e struttura del corso}

    \begin{enumerate}
        \item Introduzione a Python; elementi base di Python
        \item Elementi intermedi di Python
        \item Cenni al pensiero computazionale; realizzazione di un programma complesso
        \item Programmazione orientata ad oggetti
        \item Interazione con servizi Google
    \end{enumerate}
\end{contentframe}

\begin{contentframe}
    \frametitle{Strumenti richiesti}

    \begin{columns}
        \col{.5}
        Per lavorare in locale:
        \begin{itemize}    
            \item Interprete Python\footnote[frame]{\url{https://www.python.org/downloads/}}
            \item Editor di testo (VS Code\footnote[frame]{\url{https://code.visualstudio.com/download}})
        \end{itemize}

        \col{.5}
        Per lavorare online:
        \begin{itemize}
            \item Account Replit\footnote[frame]{\url{https://replit.com/}}
            \bigskip
        \end{itemize}
    \end{columns}
\end{contentframe}

\begin{contentframe}
    \frametitle{Conosciamoci!}
    
    Quale immagine vi rappresenta di più?

    \bigskip
    \begin{columns}
        \col{.3}
        \centering
        \image{intro1.jpg}
        
        \col{.3}
        \centering
        \image{intro2.jpg}
        
        \col{.3}
        \centering
        \image{intro3.jpg}
    \end{columns}
\end{contentframe}

\makesectionframe{Python}

\begin{exampleframe}
    \frametitle{Curiosità}

    \begin{itemize}
        \item Python non ha nulla a che fare con i rettili
        \item Prende il nome dallo show ``Monty Python's Flying Circus''
    \end{itemize}

    \bigskip
    \begin{columns}
        \col{.4}
        \centering
        \image{pythons1.jpg}
        
        \col{.4}
        \centering
        \image{pythons2.jpg}
    \end{columns}
\end{exampleframe}

\begin{contentframe}
    \frametitle{Python}

    \begin{itemize}
        \item Linguaggio di programmazione ad alto livello
        \begin{itemize}
            \item Sequenza di istruzioni che il computer esegue
            \item Si può fare qualunque cosa
            \item Alto livello: non mi devo preoccupare dei dettagli più fini
        \end{itemize}
        
        \bigskip
        \item Linguaggio interpretato
        \begin{itemize}
            \item Il codice non ha bisogno di essere compilato
        \end{itemize}

        \bigskip
        \item Esistono tanti moduli con funzionalità già fatte
        \begin{itemize}
            \item Permettono di risparmiare tempo
        \end{itemize}
    \end{itemize}
\end{contentframe}

\begin{exampleframe}
    \frametitle{Linguaggi a basso ed alto livello}

    \begin{itemize}
        \item Basso livello
        \begin{itemize}
            \item Devo indicare in dettaglio come eseguire qualsiasi azione
            \begin{itemize}
                \item \textit{Metti il valore 7 sul registro 3 della CPU, poi trasferisci il contenuto del registro 3 sulla cella 4728 della RAM} 
            \end{itemize}
            \item Esempio: Assembly
        \end{itemize}

        \bigskip
        \item Alto livello
        \begin{itemize}
            \item Molto più astratto, indico cosa fare e non come farlo
            \begin{itemize}
                \item \textit{\texttt{x = 7}}
            \end{itemize}
            \item Esempi: Python, C++, Java
        \end{itemize}
    \end{itemize}
\end{exampleframe}

\begin{exampleframe}
    \frametitle{Linguaggi compilati ed interpretati}

    \begin{itemize}
        \item Linguaggio compilato
        \begin{itemize}
            \item Tramite un programma apposito (compilatore), il codice viene convertito in istruzioni macchina
            \item Il programma risultante (.exe) si può poi eseguire direttamente senza più bisogno del compilatore
            \item Esempio: C, C++
        \end{itemize}

        \bigskip
        \item Linguaggio interpretato
        \begin{itemize}
            \item Le istruzioni sono lette ed eseguite da un programma apposito (interprete)
            \item Senza interprete, il codice non si può eseguire
            \item Esempi: Python, Java, Javascript
        \end{itemize}
    \end{itemize}
\end{exampleframe}

\begin{exerciseframe}
    \frametitle{Hello, world!}

    \begin{itemize}
        \item Eseguire il seguente programma: \fbox{\texttt{print(\textquotesingle{}Hello, world!\textquotesingle{})}}

        \bigskip
        \item Se usiamo Replit: creare un nuovo Repl di tipo Python, scrivere il codice dentro al file .py, eseguire

        \bigskip
        \item Se lavoriamo in locale: creare un file con estensione .py, scrivere il codice, invocare l'interprete col seguente comando da terminale \fbox{\texttt{python nomedelfile.py}}
    \end{itemize}
\end{exerciseframe}

\makesectionframe{Concetti base}

\begin{contentframe}
    \frametitle{Funzioni}

    \begin{itemize}
        \item Sequenza di azioni che fa una cosa

        \bigskip
        \item Esempio: \fbox{\texttt{print(...)}} stampa dati su schermo

        \bigskip
        \item Sintassi: \texttt{nomedellafunzione(argomento1, argomento2, ...)}
        \begin{itemize}
            \item Le parentesi sono obbligatorie SEMPRE
            \item Gli argomenti sono i valori su cui voglio eseguire le operazioni
        \end{itemize}
    \end{itemize}
\end{contentframe}

\begin{exerciseframe}
    \frametitle{Funzioni}

    Cosa stampano le seguenti funzioni?
    \begin{itemize}
        \item \texttt{print(1)}\pause
        \item \texttt{print(5)}\pause
        \item \texttt{print(3, 5)}\pause
        \item \texttt{print(1, 2, 3, 4, 5)}\pause
        \item \texttt{print()}\pause
        \item \texttt{print}\pause{} -- Attenzione: non viene eseguita!
    \end{itemize}
\end{exerciseframe}

\begin{contentframe}
    \frametitle{Variabili}

    \begin{columns}
        \col{.5}
        \begin{itemize}
            \item Permettono di memorizzare dati o risultati

            \bigskip
            \item Possono avere (quasi) qualunque nome
            
            \bigskip
            \item Hanno diversi tipi, in base a che dati vogliamo salvare
            \begin{itemize}
                \item Python li capisce in automatico, ma è importante saperli riconoscere
            \end{itemize}
        \end{itemize}
        
        \col{.5}
        \centering
        \image{variables.jpg}
    \end{columns}
\end{contentframe}

\begin{contentframe}
    \frametitle{Variabili}
    \framesubtitle{Numeri}

    \begin{columns}
        \col{.5}
        \centering
        {\Huge\faCalculator}\\
        \bigskip
        \texttt{int}\\
        \textbf{Numero senza virgola}\\
        \bigskip
        1\\124\\-336

        \col{.5}
        \centering
        {\Huge\faCalculator}\\
        \bigskip
        \texttt{float}\\
        \textbf{Numero con virgola}\\
        \bigskip
        -3.14\\5.0\\\bigskip
    \end{columns}
\end{contentframe}

\begin{contentframe}
    \frametitle{Variabili}
    \framesubtitle{Testo}

    \begin{columns}
        \col{1}
        \centering
        {\Huge\faPen}\\
        \bigskip
        \texttt{str}\\
        \textbf{Stringa}\\
        \bigskip
        \texttt{"}ciao\texttt{"}\\
        \texttt{"}hello\texttt{"}\\
        \texttt{"}come stai?\texttt{"}
    \end{columns}
\end{contentframe}

\begin{contentframe}
    \frametitle{Variabili}
    \framesubtitle{Logica}

    \begin{columns}
        \col{1}
        \centering
        {\Huge\faCheck}\\
        \bigskip
        \texttt{bool}\\
        \textbf{Booleano}\\
        \bigskip
        True\\False
    \end{columns}
\end{contentframe}

\begin{contentframe}
    \frametitle{Variabili}
    \framesubtitle{Strutture dati base}

    \begin{columns}
        \col{.3}
        \centering
        {\Huge\faListOl}\\
        \bigskip
        \texttt{list}\\
        \textbf{Lista}\\
        (sequenza ordinata di elementi)\\
        \bigskip
        [1, 2, 2, 3]\\
        
        [\texttt{"}ciao\texttt{"}, \texttt{"}come\texttt{"}, \texttt{"}stai\texttt{"}]

        \col{.3}
        \centering
        {\Huge\faBars}\\
        \bigskip
        \texttt{set}\\
        \textbf{Set}\\
        (sequenza senza ripetizioni e senza ordine)\\
        \bigskip
        \{1, 5, 9\}\\
        \{True, False\}

        \col{.3}
        \centering
        {\Huge\faThList}\\
        \bigskip
        \texttt{dict}\\
        \textbf{Mappa/Dizionario}\\
        (coppie chiave-valore)\\
        \bigskip
        \bigskip
        \{\textquotesingle{}Paolo\textquotesingle{}: 8, \textquotesingle{}Alice\textquotesingle{}: 9\}
        \bigskip
    \end{columns}
\end{contentframe}

\begin{contentframe}
    \frametitle{Variabili}
    \framesubtitle{Altro}

    \begin{columns}
        \col{.5}
        \centering
        {\Huge\faSitemap}\\
        \bigskip
        \texttt{class}\\
        \textbf{Classi}\\
        \bigskip
        Insieme di dati di tipologie diverse

        \col{.5}
        \centering
        {\Huge\faTimes}\\
        \bigskip
        \texttt{None}\\
        \textbf{Nessun valore}\\
        \bigskip
        \bigskip
    \end{columns}
\end{contentframe}

\begin{exerciseframe}
    \frametitle{Variabili}

    Salvare in una variabile e stampare i seguenti valori
    \begin{itemize}
        \item 7\pause
        \item 5.5\pause
        \item ``Ciao come stai?''\pause
        \item True\pause
        \item{} [0, 1, 1, 2, 3, 5, 8, 13]\pause
        \item \{0, 1, 1, 2, 3, 5, 8, 13\}\pause
        \item I voti associati a tre studenti
    \end{itemize}
\end{exerciseframe}

\begin{contentframe}
    \frametitle{Operazioni matematiche}

    \centering
    \begin{tabular}{c|c|c|c}
        Operazione                      & Operatore     & Esempio           & Risultato         \\
        \midrule
        \midrule
        Somma                           & \texttt{+}    & \texttt{4 + 3}    & \texttt{7}        \\
        Sottrazione                     & \texttt{-}    & \texttt{4 - 3}    & \texttt{1}        \\
        Moltiplicazione                 & \texttt{*}    & \texttt{4 * 3}    & \texttt{12}       \\
        Divisione                       & \texttt{/}    & \texttt{4 / 3}    & \texttt{1.3333}   \\
        \midrule
        Elevamento a potenza            & \texttt{**}   & \texttt{4 ** 3}   & \texttt{64}       \\
        Divisione senza resto           & \texttt{//}   & \texttt{4 // 3}   & \texttt{1}        \\
        Modulo (resto della divisione)  & \texttt{\%}   & \texttt{4 \% 3}   & \texttt{1}        \\
    \end{tabular}
\end{contentframe}

\begin{contentframe}
    \frametitle{Operazioni logiche}

    \centering
    \begin{tabular}{c|c|c|c}
        Operazione                      & Operatore     & Esempio                   & Risultato         \\
        \midrule
        \midrule
        Maggiore                        & \texttt{>}    & \texttt{4 > 3}            & \texttt{True}     \\
        Maggiore o uguale               & \texttt{>=}   & \texttt{4 >= 3}           & \texttt{True}     \\
        Minore                          & \texttt{<}    & \texttt{4 < 3}            & \texttt{False}    \\
        Minore o uguale                 & \texttt{<=}   & \texttt{4 <= 3}           & \texttt{False}    \\
        Uguale                          & \texttt{==}   & \texttt{4 == 3}           & \texttt{False}    \\
        Diverso                         & \texttt{!=}   & \texttt{4 != 3}           & \texttt{True}     \\
        \midrule
        Negazione                       & \texttt{not}  & \texttt{not True}         & \texttt{False}    \\
        Congiunzione                    & \texttt{and}  & \texttt{True and False}   & \texttt{False}    \\
        Disgiunzione                    & \texttt{or}   & \texttt{True or False}    & \texttt{True}     \\
    \end{tabular}
\end{contentframe}

\begin{exerciseframe}
    \frametitle{Operazioni}

    Cosa stampa questo programma?

    \bigskip
    \image[.7]{exercise_logic.jpg}
\end{exerciseframe}

\begin{contentframe}
    \frametitle{Input}

    \begin{columns}
        \col{.5}
        \begin{itemize}
            \item La funzione \fbox{\texttt{input()}} permette di leggere un valore indicato dall'utente
    
            \bigskip
            \item Come argomento opzionale posso indicare la domanda che viene chiesta prima di leggere
        \end{itemize}
        
        \col{.5}
        \centering
        \image{input.jpg}
    \end{columns}

    \bigskip
    \begin{itemize}
        \item Attenzione: il valore ritornato è sempre di tipo \texttt{str}
        \item Per leggere un numero devo convertirlo in \texttt{int}
        \begin{itemize}
            \item \fbox{\texttt{x = int(input())}}
        \end{itemize}
    \end{itemize}
\end{contentframe}

\begin{exerciseframe}
    \frametitle{Input}

    \begin{itemize}
        \item Creare un programma che legga 3 numeri e stampi il minimo, il massimo e la media dei valori inseriti
        \begin{itemize}
            \item Per calcolare minimo e massimo potete usare le funzioni \fbox{\texttt{min}} e \fbox{\texttt{max}}
        \end{itemize}
    \end{itemize}
\end{exerciseframe}

\begin{contentframe}
    \frametitle{Condizioni}

    \begin{columns}
        \col{.5}
        \begin{itemize}
            \item Permettono di eseguire delle azioni solo in certi casi
    
            \bigskip
            \item Sintassi:
            \begin{itemize}
                \item Dopo \texttt{if} c'è uno spazio
                \item La condizione ritorna un booleano (\texttt{True}/\texttt{False})
                \item Dopo la condizione ci sono i due punti (``\texttt{:}'')
                \item Le istruzioni eseguite all’interno della condizione sono indentate (con un tab o 4 spazi)
            \end{itemize}
        \end{itemize}
        
        \col{.5}
        \centering
        \image{if.jpg}
    \end{columns}
\end{contentframe}

\begin{contentframe}
    \frametitle{Condizioni}

    \begin{itemize}
        \item Se ho più condizioni, posso anche usare \texttt{else} o \texttt{elif} (\textit{else if})
    \end{itemize}

    \bigskip
    \centering
    \only<1>{
        \image{if_else.jpg}
    }
    \only<2>{
        \image[.5]{if_elif_bad.jpg}
    }
    \only<3>{
        \image[.5]{if_elif_good.jpg}
        
        \bigskip
        \texttt{elif} permette di mantenere il codice più facilmente leggibile
    }
    
\end{contentframe}

\begin{exerciseframe}
    \frametitle{Condizioni}

    \begin{itemize}
        \item Chiedere all'utente in quale città abita
        \item Se la risposta è ``Genova'', rispondere con ``Anche io!''
        \item Altrimenti, rispondere con ``Io invece abito a Genova''

        \pause
        \bigskip
        \item Accettare come risposte corrette ``Genova'', ``genova'' e ``GENOVA''
        
        \pause
        \bigskip
        \item Opzionale: accettare tutte le possibili variazioni di case
    \end{itemize}
\end{exerciseframe}

\begin{contentframe}
    \frametitle{Cicli}

    \begin{itemize}
        \item Permettono di ripetere le stesse azioni più volte

        \bigskip
        \item Due costrutti simili ed intercambiabili
        \begin{itemize}
            \item Usiamo quello più comodo per quello che dobbiamo fare
        \end{itemize}
        
        \bigskip
        \item Costrutto \texttt{while} per iterare fino a che una certa condizione non diventa falsa
        \item Costrutto \texttt{for} per iterare su una serie di valori o un certo numero di volte
    \end{itemize}
\end{contentframe}

\begin{contentframe}
    \frametitle{Cicli}
    \frametitle{While}

    \begin{columns}
        \col{.5}
        \begin{itemize}
            \item Eseguo tutte le istruzioni all'interno del ciclo fino a che la condizione è vera
            \item La condizione viene controllata prima di iniziare ogni iterazione

            \bigskip
            \item Consigliato quando ho una condizione booleana facilmente comprensibile
        \end{itemize}
        
        \col{.5}
        \centering
        \image{while.jpg}
    \end{columns}
\end{contentframe}

\begin{contentframe}
    \frametitle{Cicli}
    \frametitle{For}

    \begin{columns}
        \col{.5}
        \begin{itemize}
            \item Usato per iterare su una sequenza di valori
            \item La variabile \texttt{x} (possiamo chiamarla come vogliamo) assume ad ogni iterazione un valore diverso

            \bigskip
            \item Consigliato quando ho una sequenza di elementi
        \end{itemize}
        
        \col{.5}
        \centering
        \image{for.jpg}
    \end{columns}
\end{contentframe}

\begin{contentframe}
    \frametitle{Cicli}
    \frametitle{While vs for (1)}

    \begin{itemize}
        \item I seguenti codici sono identici, ma uno è più facile da comprendere
    \end{itemize}

    \begin{columns}
        \col{.5}
        \centering
        \image{while_example.jpg}
        
        \col{.5}
        \centering
        \image{for_example.jpg}
    \end{columns}
\end{contentframe}

\begin{contentframe}
    \frametitle{Cicli}
    \frametitle{While vs for (2)}

    \begin{itemize}
        \item I seguenti codici sono identici, ma uno è più facile da comprendere
    \end{itemize}

    \begin{columns}
        \col{.5}
        \centering
        \image{while_example2.jpg}
        
        \col{.5}
        \centering
        \image{for_example2.jpg}
    \end{columns}
\end{contentframe}

\begin{exerciseframe}
    \frametitle{Cicli}

    \begin{itemize}
        \item Chiedere un numero $x$ a scelta dell'utente e stampare la tabellina di quel numero (da $x*0$ a $x*10$)

        \pause
        \bigskip
        \item Chiedere all'utente due numeri ($x$ e $n$) e stampare tutti i multipli di $x$ minori o uguali ad $n$
    \end{itemize}
\end{exerciseframe}

\begin{exerciseframe}
    \frametitle{Cicli e strutture dati (1)}

    \begin{itemize}
        \item Permettere all'utente di inserire numeri a sua scelta, fino a che inserisce $0$. Stampare infine tutti i numeri inseriti nello stesso ordine
        
        \pause
        \bigskip
        \item Stampare tutti i numeri inseriti in ordine inverso
        \begin{itemize}
            \item Suggerimento: usare le funzioni \fbox{\texttt{range}} e \fbox{\texttt{len}}
        \end{itemize}

        \pause
        \bigskip
        \item Stampare il numero di valori inseriti, il minimo, il massimo, la media
        \begin{itemize}
            \item Funzioni consigliate: \fbox{\texttt{len}}, \fbox{\texttt{min}}, \fbox{\texttt{max}}, \fbox{\texttt{sum}}
        \end{itemize}
    \end{itemize}
\end{exerciseframe}

\begin{exerciseframe}
    \frametitle{Cicli e strutture dati (2)}

    \begin{itemize}
        \item Permettere all'utente di inserire numeri a sua scelta, fino a che inserisce $0$. Stampare infine tutti i numeri inseriti in ordine crescente senza duplicati
    \end{itemize}
\end{exerciseframe}

\makesectionframe{Elementi intermedi}

\begin{contentframe}
    \frametitle{File}

    \begin{columns}
        \col{.5}
        \begin{itemize}
            \item Due operazioni possibili
            \begin{itemize}
                \item Lettura
                \item Scrittura
            \end{itemize}
    
            \bigskip
            \item Processo per qualsiasi operazione:
            \begin{itemize}
                \item Aprire il file
                \item Fare le operazioni necessarie
                \item Chiudere il file
            \end{itemize}
        \end{itemize}
        
        \col{.5}
        \centering
        \image{file.jpg}
    \end{columns}
\end{contentframe}

\begin{contentframe}
    \frametitle{File}
    \framesubtitle{Keyword \texttt{with}}

    \begin{itemize}
        \item Con la keywork \texttt{with}, Python chiude il file in automatico
    \end{itemize}

    \bigskip
    \begin{columns}
        \col{.5}
        \image{file.jpg}
        
        \col{.5}
        \centering
        \image{file_with.jpg}
    \end{columns}
\end{contentframe}

\begin{contentframe}
    \frametitle{File}
    \framesubtitle{Modalità di apertura file}

    Il secondo argomento indica la modalità di apertura:
    
    \begin{columns}
        \col{.5}
        \begin{itemize}
            \item \texttt{r} \textit{(read)}
            \begin{itemize}
                \item Solo lettura (default)
            \end{itemize}
            \item \texttt{w} \textit{(write)}
            \begin{itemize}
                \item Solo scrittura
                \item Cancella il contenuto precedente
            \end{itemize}
        \end{itemize}
    
        \col{.5}
        \begin{itemize}
            \item \texttt{a} \textit{(append)}
            \begin{itemize}
                \item Solo scrittura
                \item Mantiene il contenuto precedente
            \end{itemize}
            \item \texttt{w+} \textit{(read-write)}
            \begin{itemize}
                \item Sia lettura che scrittura
            \end{itemize}
        \end{itemize}
    \end{columns}
\end{contentframe}

\begin{exerciseframe}
    \frametitle{File}

    \begin{itemize}
        \item Leggere il contenuto di un file e stampare la prima riga su schermo
        \bigskip
        \item Creare un file chiamato \texttt{prova.txt} che contenga ``Hello, world!''
        \bigskip
        \item Chiedere all'utente un numero e scriverlo in fondo al file \texttt{prova.txt}. Il contenuto precendente del file deve rimanere invariato
    \end{itemize}
\end{exerciseframe}

\begin{contentframe}
    \frametitle{Librerie}

    \begin{itemize}
        \item Insieme di funzioni già pronte scritte da altre persone
        \bigskip
        \item Esistono numerose librerie
        \item Ciascuna si occupa (generalmente) di una sola cosa
        \bigskip
        \item Per usarle nel mio codice uso il comando \texttt{import}
    \end{itemize}
\end{contentframe}

\begin{contentframe}
    \frametitle{Librerie}
    \framesubtitle{random}

    \begin{columns}
        \col{.5}
        \begin{itemize}
            \item Permette di generare numeri casuali
        \end{itemize}
        
        \col{.5}
        \centering
        \image{random.jpg}
    \end{columns}
\end{contentframe}

\begin{contentframe}
    \frametitle{Librerie}
    \framesubtitle{time}

    \begin{columns}
        \col{.5}
        \begin{itemize}
            \item Operazioni relative al tempo
            \bigskip
            \item \texttt{sleep} permette di bloccare l'esecuzione del programma per un certo numero di secondi
        \end{itemize}
        
        \col{.5}
        \centering
        \image{time.jpg}
    \end{columns}
\end{contentframe}

\begin{contentframe}
    \frametitle{Librerie}
    \framesubtitle{datetime}

    \begin{columns}
        \col{.5}
        \begin{itemize}
            \item Operazioni su date e ore
            \bigskip
            \item Contiene:
            \begin{itemize}
                \item \texttt{datetime.date}
                \item \texttt{datetime.time}
                \item \texttt{datetime.datetime}
            \end{itemize}
            \bigskip
            \item \texttt{datetime.datetime.now} ritorna il timestamp corrente 
        \end{itemize}
        
        \col{.5}
        \centering
        \image{datetime.jpg}
    \end{columns}
\end{contentframe}

\begin{exerciseframe}
    \frametitle{Librerie}

    \begin{itemize}
        \item Stampare un numero casuale tra 1 e 22. Eseguire il programma più volte per assicurarsi del suo corretto funzionamento.

        \pause
        \bigskip
        \item Stampare la data e l'ora corrente.

        \pause
        \bigskip
        \item Creare un programma che stampa l'ora attuale, dorme (\texttt{time.sleep}) per 2 secondi e stampa l'ora a cui si risveglia.
        \pause
        \begin{itemize}
            \item Calcolare la quantità di tempo effettivamente trascorsa: sono esattamente 2 secondi?
        \end{itemize}
        
    \end{itemize}
\end{exerciseframe}

\begin{contentframe}
    \frametitle{Funzioni}

    \begin{columns}
        \col{.5}
        \begin{itemize}
            \item Blocchi di codice che fanno una cosa
            \item Posso riutilizzarli senza dover riscrivere tutto il codice ogni volta
            \item Permettono di organizzare meglio il codice
        \end{itemize}
        
        \col{.5}
        \centering
        \image{def.jpg}
    \end{columns}
\end{contentframe}

\begin{contentframe}
    \frametitle{Funzioni}
    \framesubtitle{Sintassi}

    \begin{columns}
        \col{.5}
        \begin{itemize}
            \item \texttt{def} indica che voglio definire una nuova funzione
            \bigskip
            \item \texttt{return} permette alla funzione di ritornare un risultato
            \begin{itemize}
                \item \texttt{return} interrompe l'esecuzione della funzione!
            \end{itemize}
            \bigskip
            \item Posso passare dei valori tramite argomenti
        \end{itemize}
        
        \col{.5}
        \centering
        \image{def.jpg}
    \end{columns}
\end{contentframe}

\begin{contentframe}
    \frametitle{Funzioni}
    \framesubtitle{Ricorsione}

    \begin{columns}
        \col{.5}
        \begin{itemize}
            \item Una funzione può a sua volta chiamare altre funzioni
            \bigskip
            \item Se una funzione chiama se stessa, si dice \textit{ricorsiva}
            \begin{itemize}
                \item Generalmente gli argomenti sono diversi
            \end{itemize}
        \end{itemize}
        
        \col{.5}
        \centering
        \image{def_fib.jpg}
    \end{columns}
\end{contentframe}

\begin{exerciseframe}
    \frametitle{Esercizi}

    \begin{enumerate}
        \item Creare una funzione che converte una temperatura Fahrenheit in Celsius
        \begin{itemize}
            \item $C = \frac{(F - 32) * 5}{9}$
        \end{itemize}
        \pause
        \bigskip
        \item Creare una funzione che ritorna \texttt{True} se un numero è pari, altrimenti \texttt{False}
        \pause
        \bigskip
        \item Creare una funzione che, con l'aiuto della funzione creata precedentemente, indica quanti numeri sono pari in una lista di numeri
        \pause
        \bigskip
        \item Creare una funzione che conta il numero di vocali in una stringa fornita dall'utente
    \end{enumerate}
\end{exerciseframe}

\makesectionframe{Pensiero computazionale}

\begin{contentframe}
    \frametitle{Pensiero computazionale}

    \begin{itemize}
        \item Abilità di pensare in modo simile ai computer
        \bigskip
        \item Aiuta a scrivere programmi complessi ed organizzare meglio il codice
        \bigskip
        \item Ci concentreremo sulla decomposizione di un problema in sottoproblemi
    \end{itemize}
\end{contentframe}


\begin{contentframe}
    \frametitle{Divide et impera}

    \begin{itemize}
        \item Un qualsiasi problema si può definire come un insieme di sottoproblemi più semplici
        \item Risolvere una parte di un problema è più facile che risolvere l'intero problema in un colpo solo
        \bigskip
        \item Questo processo si ripete fino a che ogni parte è semplicissima da risolvere
    \end{itemize}
\end{contentframe}

\begin{contentframe}
    \frametitle{Divide et impera}
    \framesubtitle{Problema generale}

    \centering
    \smartarttree[Preparare un caffè con la moka]
\end{contentframe}


\begin{contentframe}
    \frametitle{Divide et impera}
    \framesubtitle{Problema generale}

    \centering
    \smartarttree[
        Preparare un caffè con la moka
        [Preparare la moka]
        [Far venire su il caffè]
    ]
\end{contentframe}

\begin{contentframe}
    \frametitle{Divide et impera}
    \framesubtitle{Problema generale}

    \centering
    \smartarttree[
        Preparare un caffè con la moka
        [Preparare la moka]
        [Far venire su il caffè
            [Accendere il fuoco]
            [Aspettare che il caffè venga su]
            [Spegnere il fuoco]
        ]
    ]
\end{contentframe}

\begin{contentframe}
    \frametitle{Divide et impera}
    \framesubtitle{Problema di programmazione}

    \centering
    \smartarttree[
        Contare il numero di parole diverse\\presenti in tutti i file del computer 
    ]
\end{contentframe}

\begin{contentframe}
    \frametitle{Divide et impera}
    \framesubtitle{Problema di programmazione}

    \centering
    \smartarttree[
        Contare il numero di parole diverse\\presenti in tutti i file del computer 
        [Cercare tutti i\\file del computer
        ]
        [Contare le parole\\diverse in un file
        ]
    ]
\end{contentframe}

\begin{contentframe}
    \frametitle{Divide et impera}
    \framesubtitle{Problema di programmazione}

    \centering
    \smartarttree[
        Contare il numero di parole diverse\\presenti in tutti i file del computer 
        [Cercare tutti i\\file del computer
            [Cercare\\nella cartella\\corrente]
            [Cercare\\nelle\\sottocartelle]
        ]
        [Contare le parole\\diverse in un file
        ]
    ]
\end{contentframe}

\begin{contentframe}
    \frametitle{Divide et impera}
    \framesubtitle{Problema di programmazione}

    \centering
    \smartarttree[
        Contare il numero di parole diverse\\presenti in tutti i file del computer 
        [Cercare tutti i\\file del computer
            [Cercare\\nella cartella\\corrente]
            [Cercare\\nelle\\sottocartelle]
        ]
        [Contare le parole\\diverse in un file
            [Leggere il\\contenuto\\del file]
            [Separare\\le\\parole]
            [Aggiungere la\\parola se non\\è già presente]
        ]
    ]
\end{contentframe}

\begin{exampleframe}
    \frametitle{Divide et impera}

    \begin{itemize}
        \item Vediamo un esempio pratico
        \bigskip
        \item Vogliamo creare un programma che faccia le seguenti azioni...
    \end{itemize}
\end{exampleframe}

\begin{exampleframe}
    \frametitle{Divide et impera}

    \begin{enumerate}
        \item Chiedere all’utente il nome di un file
        \begin{itemize}
            \item Il nome non può essere vuoto e non può contenere i caratteri `\texttt{?}', `\texttt{/}' e `\texttt{*}'
            \item Se il nome non è valido, va richiesto fino a che non sia corretto
        \end{itemize}
        \item Aprire il file e leggere il suo contenuto
        \item Contare il numero di vocali contenute nel file
        \item Chiedere all’utente il nome di un secondo file
        \begin{itemize}
            \item Come prima, se il nome non è corretto, richiederlo
        \end{itemize}
        \item Aprire il secondo file e scrivere il conteggio effettuato in precedenza
    \end{enumerate}
\end{exampleframe}

\begin{exampleframe}
    \frametitle{Divide et impera}
    \framesubtitle{Una possibile soluzione}

    \centering
    \image[1][.7]{functions_bad.jpg}
\end{exampleframe}

\begin{exampleframe}
    \frametitle{Divide et impera}
    \framesubtitle{Creaiamo delle funzioni}

    \centering
    \image[1][.7]{functions1.jpg}
\end{exampleframe}

\begin{exampleframe}
    \frametitle{Divide et impera}
    \framesubtitle{Codice finale}

    \centering
    \image[1][.7]{functions2.jpg}
\end{exampleframe}

\begin{exerciseframe}
    \frametitle{Esercizio finale}

    Creare un programma che faccia le seguenti azioni.
    Organizzare il codice in funzioni come spiegato.

    \begin{enumerate}
        \item Chiedere all’utente il nome di un file
        \begin{itemize}
            \item Il nome non può essere vuoto e non può contenere i caratteri `\texttt{?}', `\texttt{/}' e `\texttt{*}'
            \item Se il nome non è valido, va richiesto fino a che non sia corretto
        \end{itemize}
        \item Aprire il file e leggere il suo contenuto
        \item Contare il numero di vocali contenute nel file
        \item Chiedere all’utente il nome di un secondo file
        \begin{itemize}
            \item Come prima, se il nome non è corretto, richiederlo
        \end{itemize}
        \item Aprire il secondo file e scrivere il conteggio effettuato in precedenza
    \end{enumerate}
\end{exerciseframe}

\makesectionframe{Object-Oriented Programming}[Programmazione orientata ad oggetti]

\begin{contentframe}
    \frametitle{Object-Oriented Programming}

    \begin{itemize}
        \item Tecnica di programmazione
        \begin{itemize}
            \item Stesso linguaggio
            \item Stessi programmi
            \item Diverso metodo di organizzare il codice
        \end{itemize}
    \end{itemize}
\end{contentframe}

\begin{contentframe}
    \frametitle{Object-Oriented Programming}

    \begin{itemize}
        \item Ogni ``entità'' è vista come un \textbf{oggetto}

        \bigskip
        \item Su quell'oggetto applico diverse operazioni ed ottengo un altro oggetto
        \item Ripeto fino ad arrivare al risultato

        \bigskip
        \item I programmi non sono più una sequenza di azioni, ma un insieme di oggetti che interagiscono tra di loro
    \end{itemize}
\end{contentframe}

\begin{contentframe}
    \frametitle{Perché?}
    \framesubtitle{Semplicità concettuale}

    \begin{columns}
        \col{.5}
        \textbf{Senza oggetti}
        \begin{itemize}
            \item Devo tenere a mente l'intera situazione in ogni momento
        \end{itemize}
        
        \col{.5}
        \textbf{Con oggetti}
        \begin{itemize}
            \item Tengo a mente solo una parte del problema alla volta
        \end{itemize}
    \end{columns}
\end{contentframe}

\begin{contentframe}
    \frametitle{Perché?}
    \framesubtitle{Organizzazione codice}

    \begin{columns}
        \col{.5}
        \textbf{Senza oggetti}
        \begin{itemize}
            \item Ho molte funzioni diverse nello stesso blocco di codice
        \end{itemize}
        
        \col{.5}
        \textbf{Con oggetti}
        \begin{itemize}
            \item In un certo blocco di codice ho solo funzioni relative ad un singolo concetto
        \end{itemize}
    \end{columns}
\end{contentframe}

\begin{contentframe}
    \frametitle{Perché?}
    \framesubtitle{Semplicità programmazione}

    \begin{columns}
        \col{.5}
        \textbf{Senza oggetti}
        \begin{itemize}
            \item Ho a disposizione in ogni momento tutte le funzioni esistenti
        \end{itemize}
        
        \col{.5}
        \textbf{Con oggetti}
        \begin{itemize}
            \item Ho a disposizione solo le funzione rilevanti ai dati che sto usando
        \end{itemize}
    \end{columns}
\end{contentframe}

\makesectionframe{Classi}

\begin{contentframe}
    \frametitle{Classi}

    \begin{columns}
        \col{.5}
        \begin{itemize}
            \item Usate per rapprentare gli oggetti
            \begin{itemize}
                \item Secondo la OOP, i programmi sono classi che interagiscono tra di loro
            \end{itemize}

            \bigskip
            \item Classe
            \begin{itemize}
                \item Insieme di dati (e funzioni)
                \item Relativa ad un singolo concetto
            \end{itemize}
        \end{itemize}
        
        \col{.5}
        \centering
        \image{class.jpg}
        {
            \footnotesize
            \textit{Questo è un esempio, la sintassi reale è leggermente diversa}
        }
    \end{columns}
\end{contentframe}


\begin{contentframe}
    \frametitle{Metodi}

    \begin{columns}
        \col{.5}
        \begin{itemize}
            \item Funzioni contenute dentro ad una classe che operano sui dati di quella classe

            \bigskip
            \item Per accedere ai dati della classe, devo usare \texttt{self}
            \begin{itemize}
                \item Salvo rare eccezioni, \texttt{self} è il primo argomento di ogni metodo
            \end{itemize}
        \end{itemize}
        
        \col{.5}
        \centering
        \image{class_method.jpg}
    \end{columns}
\end{contentframe}

\begin{contentframe}
    \frametitle{Metodi}

    \begin{columns}
        \col{.5}
        \begin{itemize}
            \item Nel programma, quando ho una persona, ho a dispozione anche un metodo per controllare se è minorenne

            \bigskip
            \item Per accedere agli elementi contenuti in un oggetto, uso il `\texttt{.}' \textit{(dot-notation)}
        \end{itemize}
        
        \col{.5}
        \centering
        \image{class_method_usage.jpg}
    \end{columns}
\end{contentframe}

\begin{contentframe}
    \frametitle{Costruttore}

    \begin{columns}
        \col{.5}
        \begin{itemize}
            \item Metodo per inizializzare una classe
            \item Posso anche passare dei valori

            \bigskip
            \item La funzione si chiama \texttt{\_\_init\_\_}
        \end{itemize}
        
        \col{.5}
        \centering
        \image{class_init.jpg}
    \end{columns}
\end{contentframe}

\begin{contentframe}
    \frametitle{Costruttore}

    \begin{columns}
        \col{.5}
        \begin{itemize}
            \item Nel programma principale, passo i valori quando creo l'oggetto
        \end{itemize}
        
        \col{.5}
        \centering
        \image{class_init_usage.jpg}
    \end{columns}
\end{contentframe}

\begin{exerciseframe}
    \frametitle{Esercizio (1/2)}

    \begin{itemize}
        \item Creare la classe \texttt{Item}, che contiene:
        \begin{itemize}
            \item \texttt{name} (Nome dell'articolo)
            \item \texttt{price} (Prezzo dell'articolo)
        \end{itemize}

        \bigskip
        \item Creare la classe \texttt{Person}, che contiene:
        \begin{itemize}
            \item \texttt{name} (Nome della persona)
            \item \texttt{money} (Soldi a disposizione della persona)
            \item \texttt{items} (Oggetti posseduti dalla persona. Al momento della creazione: nessuno)
            \item \texttt{buy(self, item)} (Metodo che acquista un oggetto, se la persona se lo può permettere)
        \end{itemize}
    \end{itemize}
\end{exerciseframe}

\begin{exerciseframe}
    \frametitle{Esercizio (2/2)}

    \begin{itemize}
        \item Nel programma principale, eseguire le seguenti azioni:
        \begin{enumerate}
            \item Creare la persona `Mario', che ha a disposizione 25€
            \item Creare una mela da 7€
            \item Creare una pera da 5€
            \item Creare un limone da 25€
            \item Creare un arancio da 5€
            \item Far acquistare a Mario una mela, due pere, un limone e tre arance
            \item Stampare quanti soldi ha Mario
            \item Stampare gli articoli che Mario possiede
            \pause
            \bigskip
            \item Provare ad eseguire queste azioni con più persone
        \end{enumerate}
    \end{itemize}
\end{exerciseframe}

\makesectionframe{Google Colab}

\begin{contentframe}
    \frametitle{Google Colab}

    \begin{itemize}
        \item Strumento per scrivere ed eseguire codice Python online\footnote[frame]{\url{https://colab.research.google.com/}}
        \bigskip
        \item Permette di interagire facilmente con i servizi Google Drive
    \end{itemize}
\end{contentframe}

\begin{exampleframe}
    \frametitle{Alcui moduli utili}

    \begin{itemize}
        \item \texttt{from google.colab import drive}\\
            \texttt{drive.mount(\textquotesingle{}/content/drive\textquotesingle{})}
        \begin{itemize}
            \item Permette di interagire con i file salvati su Google Drive
        \end{itemize}

        \bigskip
        \item Modulo \texttt{pandas}
        \begin{itemize}
            \item Operazioni su file CSV e Excel
        \end{itemize}

        \bigskip
        \item Per informazioni, consultare internet! 
    \end{itemize}
\end{exampleframe}

\begin{exerciseframe}
    \frametitle{Esercizio guidato}

    Abbiamo un file CSV su Google Drive con le medie degli studenti in diverse materie.
    
    Vogliamo scoprire:
    \begin{itemize}
        \item La media dei voti per ciascuna materia
        \item La media complessiva per ciascuno studente
        \item L'esito finale per ciascuno studente
        \begin{itemize}
            \item \textit{(se la $media \geq 6$ lo studente è promosso, se no è bocciato)}
        \end{itemize}
    \end{itemize}
\end{exerciseframe}

\begin{exerciseframe}
    \frametitle{Esercizio finale}

    \begin{itemize}
        \item A partire dall'esercizio precedente, generare tramite Python dei grafici per mostrare meglio le informazioni contenute nel dataset

    \bigskip
    \item Consigli:
        \begin{itemize}
            \item Usare internet per cercare informazioni, esempi di codice o domande fatte da altri utenti
            \item Non aver paura di provare a fare cose, anche (quasi) a caso
        \end{itemize}
    \end{itemize}
\end{exerciseframe}

\makesectionframe{Grazie per l'attenzione}

\begin{exampleframe}
    \frametitle{Questionario finale}

    \begin{itemize}
        \item \url{https://forms.gle/fDLgiFuuhatEvGW16}
    \end{itemize}
\end{exampleframe}
