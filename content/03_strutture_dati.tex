

\begin{exerciseframe}
    \frametitle{Cicli e strutture dati (1)}

    \begin{itemize}
        \item Permettere all'utente di inserire numeri a sua scelta, fino a che inserisce $0$. Stampare infine tutti i numeri inseriti nello stesso ordine
        
        \pause
        \bigskip
        \item Stampare tutti i numeri inseriti in ordine inverso
        \begin{itemize}
            \item Suggerimento: usare le funzioni \fbox{\texttt{range}} e \fbox{\texttt{len}}
        \end{itemize}

        \pause
        \bigskip
        \item Stampare il numero di valori inseriti, il minimo, il massimo, la media
        \begin{itemize}
            \item Funzioni consigliate: \fbox{\texttt{len}}, \fbox{\texttt{min}}, \fbox{\texttt{max}}, \fbox{\texttt{sum}}
        \end{itemize}
    \end{itemize}
\end{exerciseframe}

\begin{exerciseframe}
    \frametitle{Cicli e strutture dati (2)}

    \begin{itemize}
        \item Permettere all'utente di inserire numeri a sua scelta, fino a che inserisce $0$. Stampare infine tutti i numeri inseriti in ordine crescente senza duplicati
    \end{itemize}
\end{exerciseframe}

