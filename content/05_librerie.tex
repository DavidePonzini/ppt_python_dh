\begin{contentframe}
    \frametitle{Librerie}

    \begin{itemize}
        \item Insieme di funzioni già pronte scritte da altre persone
        \bigskip
        \item Esistono numerose librerie
        \item Ciascuna si occupa (generalmente) di una sola cosa
        \bigskip
        \item Per usarle nel mio codice uso il comando \texttt{import}
    \end{itemize}
\end{contentframe}

\begin{contentframe}
    \frametitle{Librerie}
    \framesubtitle{random}

    \begin{columns}
        \col{.5}
        \begin{itemize}
            \item Permette di generare numeri casuali
        \end{itemize}
        
        \col{.5}
        \centering
        \image{random.jpg}
    \end{columns}
\end{contentframe}

\begin{contentframe}
    \frametitle{Librerie}
    \framesubtitle{time}

    \begin{columns}
        \col{.5}
        \begin{itemize}
            \item Operazioni relative al tempo
            \bigskip
            \item \texttt{sleep} permette di bloccare l'esecuzione del programma per un certo numero di secondi
        \end{itemize}
        
        \col{.5}
        \centering
        \image{time.jpg}
    \end{columns}
\end{contentframe}

\begin{contentframe}
    \frametitle{Librerie}
    \framesubtitle{datetime}

    \begin{columns}
        \col{.5}
        \begin{itemize}
            \item Operazioni su date e ore
            \bigskip
            \item Contiene:
            \begin{itemize}
                \item \texttt{datetime.date}
                \item \texttt{datetime.time}
                \item \texttt{datetime.datetime}
            \end{itemize}
            \bigskip
            \item \texttt{datetime.datetime.now} ritorna il timestamp corrente 
        \end{itemize}
        
        \col{.5}
        \centering
        \image{datetime.jpg}
    \end{columns}
\end{contentframe}

\begin{exerciseframe}
    \frametitle{Librerie}

    \begin{itemize}
        \item Stampare un numero casuale tra 1 e 22. Eseguire il programma più volte per assicurarsi del suo corretto funzionamento.

        \pause
        \bigskip
        \item Stampare la data e l'ora corrente.

        \pause
        \bigskip
        \item Creare un programma che stampa l'ora attuale, dorme (\texttt{time.sleep}) per 2 secondi e stampa l'ora a cui si risveglia.
        \pause
        \begin{itemize}
            \item Calcolare la quantità di tempo effettivamente trascorsa: sono esattamente 2 secondi?
        \end{itemize}
        
    \end{itemize}
\end{exerciseframe}
