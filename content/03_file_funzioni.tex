\makesectionframe{File e funzioni}


\begin{contentframe}
    \frametitle{File}

    \begin{columns}
        \col{.5}
        \begin{itemize}
            \item Due operazioni possibili
            \begin{itemize}
                \item Lettura
                \item Scrittura
            \end{itemize}
    
            \bigskip
            \item Processo per qualsiasi operazione:
            \begin{itemize}
                \item Aprire il file
                \item Fare le operazioni necessarie
                \item Chiudere il file
            \end{itemize}
        \end{itemize}
        
        \col{.5}
        \centering
        \image{file.jpg}
    \end{columns}
\end{contentframe}

\begin{contentframe}
    \frametitle{File}
    \framesubtitle{Keyword \texttt{with}}

    \begin{itemize}
        \item Con la keywork \texttt{with}, Python chiude il file in automatico
    \end{itemize}

    \bigskip
    \begin{columns}
        \col{.5}
        \image{file.jpg}
        
        \col{.5}
        \centering
        \image{file_with.jpg}
    \end{columns}
\end{contentframe}

\begin{contentframe}
    \frametitle{File}
    \framesubtitle{Modalità di apertura file}

    Il secondo argomento indica la modalità di apertura:
    
    \begin{columns}
        \col{.5}
        \begin{itemize}
            \item \texttt{r} \textit{(read)}
            \begin{itemize}
                \item Solo lettura (default)
            \end{itemize}
            \item \texttt{w} \textit{(write)}
            \begin{itemize}
                \item Solo scrittura
                \item Cancella il contenuto precedente
            \end{itemize}
        \end{itemize}
    
        \col{.5}
        \begin{itemize}
            \item \texttt{a} \textit{(append)}
            \begin{itemize}
                \item Solo scrittura
                \item Mantiene il contenuto precedente
            \end{itemize}
            \item \texttt{w+} \textit{(read-write)}
            \begin{itemize}
                \item Sia lettura che scrittura
            \end{itemize}
        \end{itemize}
    \end{columns}
\end{contentframe}

\begin{exerciseframe}
    \frametitle{Esercizi (1)}
    \framesubtitle{File}

    \begin{enumerate}
        \item Leggere il contenuto di un file e stampare la prima riga su schermo
        \bigskip
        \item Creare un file chiamato \texttt{prova.txt} che contenga ``Hello, world!''
        \bigskip
        \item Chiedere all'utente un numero e scriverlo in fondo al file \texttt{prova.txt}. Il contenuto precedente del file deve rimanere invariato
    \end{enumerate}
\end{exerciseframe}

\begin{exerciseframe}
    \frametitle{Esercizi (2)}
    \framesubtitle{File}

    Scaricare il seguente \href{https://raw.githubusercontent.com/DavidePonzini/didattica/refs/heads/main/moby_dick.txt}{file} e salvarlo sul vostro PC
    
    \begin{enumerate}
        \setcounter{enumi}{3}
        \item Contare il numero di caratteri presenti nel file
        \item Contare il numero di parole presenti nel file \textit{(suggerimento: usare la funzione \texttt{split})}
        \item Contare il numero di parole diverse presenti nel file
        \item Contare il numero di occorrenze di ciascuna parola presente nel file
    \end{enumerate}
\end{exerciseframe}


\begin{contentframe}
    \frametitle{Funzioni}

    \begin{columns}
        \col{.5}
        \begin{itemize}
            \item Blocchi di codice che fanno una cosa
            \item Posso riutilizzarli senza dover riscrivere tutto il codice ogni volta
            \item Permettono di organizzare meglio il codice
        \end{itemize}
        
        \col{.5}
        \centering
        \image{def.jpg}
    \end{columns}
\end{contentframe}

\begin{contentframe}
    \frametitle{Funzioni}
    \framesubtitle{Sintassi}

    \begin{columns}
        \col{.5}
        \begin{itemize}
            \item \texttt{def} indica che voglio definire una nuova funzione
            \bigskip
            \item \texttt{return} permette alla funzione di ritornare un risultato
            \begin{itemize}
                \item \texttt{return} interrompe l'esecuzione della funzione!
            \end{itemize}
            \bigskip
            \item Posso passare dei valori tramite argomenti
        \end{itemize}
        
        \col{.5}
        \centering
        \image{def.jpg}
    \end{columns}
\end{contentframe}


\begin{contentframe}
    \frametitle{Funzioni}
    \framesubtitle{Argomenti delle funzioni}

    \begin{itemize}
        \item Una funzione può avere più argomenti
        \begin{itemize}
            \item Sono identificati dal loro ordine
        \end{itemize}

        \bigskip
        \item Gli argomenti possono avere dei valori di default, nel caso non siano specificati
        \begin{itemize}
            \item Questi argomenti sono definiti ``opzionali''
        \end{itemize}

        \bigskip
        \item Gli argomenti possono essere specificati per posizione o per nome
    \end{itemize}
\end{contentframe}

\begin{contentframe}
    \frametitle{Funzioni}
    \framesubtitle{Argomenti delle funzioni}

    \centering
    \image[1][.7]{def_args.jpg}
\end{contentframe}

\begin{contentframe}
    \frametitle{Funzioni}
    \framesubtitle{Argomenti delle funzioni (2)}

    \begin{itemize}
        \item Esistono due argomenti speciali, indicati con \texttt{*} e \texttt{**}
        
        \bigskip
        \item \texttt{*args} prende un numero illimitato di argomenti posizionali e li salva come una lista

        \bigskip
        \item \texttt{*kwargs} prende un numero illimitato di argomenti opzionali e li salva come un dizionario

        \bigskip
        \item Si usano solo in certe situazioni, ma sono usati da funzioni molto comuni 
    \end{itemize}
\end{contentframe}

\begin{exampleframe}
    \frametitle{Argomenti delle funzioni}
    \framesubtitle{Esempio}

    \begin{itemize}
        \item La funzione \texttt{print} ha i seguenti argomenti:\\
            \boxed{\texttt{def print(*args, sep=\textquotesingle{}~\textquotesingle{}, end=\textquotesingle{}\\n\textquotesingle{}, file=None):}}

        \begin{itemize}
            \item \texttt{*args} prende i valori che volete stampare
            \item \texttt{sep} (opzionale) indica il separatore tra un valore e l'altro
            \item \texttt{end} (opzionale) indica cosa verrà stampato dopo l'ultimo valore
            \item \texttt{file} (opzionale) indica dove verrà stampato il testo. Se è \texttt{None}, stampa su schermo
        \end{itemize}
    \end{itemize}
\end{exampleframe}

\begin{contentframe}
    \frametitle{Funzioni}
    \framesubtitle{Ricorsione}

    \begin{columns}
        \col{.5}
        \begin{itemize}
            \item Una funzione può a sua volta chiamare altre funzioni
            \bigskip
            \item Se una funzione chiama se stessa, si dice \textit{ricorsiva}
            \begin{itemize}
                \item Generalmente gli argomenti sono diversi
                \item In certe situazioni permette di fare azioni complesse con poche righe di codice
            \end{itemize}
        \end{itemize}
        
        \col{.5}
        \centering
        \image{def_fib.jpg}
    \end{columns}
\end{contentframe}

\begin{exerciseframe}
    \frametitle{Esercizi}

    \begin{enumerate}
        \item Creare una funzione che converte una temperatura Fahrenheit in Celsius
        \begin{itemize}
            \item $C = \frac{(F - 32) * 5}{9}$
        \end{itemize}
        \pause
        \bigskip
        \item Creare una funzione che ritorna \texttt{True} se un numero è pari, altrimenti \texttt{False}
        \pause
        \bigskip
        \item Creare una funzione che, con l'aiuto della funzione creata precedentemente, indica quanti numeri sono pari in una lista di numeri
        \pause
        \bigskip
        \item Creare una funzione che conta il numero di vocali in una stringa fornita dall'utente
    \end{enumerate}
\end{exerciseframe}

% \begin{exerciseframe}
%     \frametitle{Esercizio più complesso}

%     \begin{enumerate}
%         \item Creare una funzione \boxed{\texttt{vocale(lettera)}} che restituisce \texttt{True} se \texttt{lettera} è una vocale, altrimenti ritorna \texttt{False}

%         \pause
%         \bigskip
%         \item Creare una funzione \boxed{\texttt{conta\_vocali(parola)}} che restituisce il numero di vocali in una parola
        
%         \pause
%         \bigskip
%         \item Creare una funzione \boxed{\texttt{separa(testo)}} che, data una frase, ritorna una lista contenente le varie parole separate tra loro \textit{(suggerimento: usare la funzione   \texttt{split()})}

%         \pause
%         \bigskip
%         \item Creare un programma che, dato un testo, calcola il numero massimo di vocali trovate in una parola
%     \end{enumerate}
% \end{exerciseframe}

\makesectionframe{Pensiero computazionale}

\begin{contentframe}
    \frametitle{Pensiero computazionale}

    \begin{itemize}
        \item Abilità di pensare in modo simile ai computer
        \bigskip
        \item Aiuta a scrivere programmi complessi ed organizzare meglio il codice
        \bigskip
        \item Ci concentreremo sulla decomposizione di un problema in sottoproblemi
    \end{itemize}
\end{contentframe}


\begin{contentframe}
    \frametitle{Divide et impera}

    \begin{itemize}
        \item Un qualsiasi problema si può definire come un insieme di sottoproblemi più semplici
        \item Risolvere una parte di un problema è più facile che risolvere l'intero problema in un colpo solo
        \bigskip
        \item Questo processo si ripete fino a che ogni parte è semplicissima da risolvere
    \end{itemize}
\end{contentframe}

\begin{contentframe}
    \frametitle{Divide et impera}
    \framesubtitle{Problema generale}

    \centering
    \smartarttree[Preparare un caffè con la moka]
\end{contentframe}


\begin{contentframe}
    \frametitle{Divide et impera}
    \framesubtitle{Problema generale}

    \centering
    \smartarttree[
        Preparare un caffè con la moka
        [Preparare la moka]
        [Far venire su il caffè]
    ]
\end{contentframe}

\begin{contentframe}
    \frametitle{Divide et impera}
    \framesubtitle{Problema generale}

    \centering
    \smartarttree[
        Preparare un caffè con la moka
        [Preparare la moka]
        [Far venire su il caffè
            [Accendere il fuoco]
            [Aspettare che il caffè venga su]
            [Spegnere il fuoco]
        ]
    ]
\end{contentframe}

\begin{contentframe}
    \frametitle{Divide et impera}
    \framesubtitle{Problema di programmazione}

    \centering
    \smartarttree[
        Contare il numero di parole diverse\\presenti in tutti i file del computer 
    ]
\end{contentframe}

\begin{contentframe}
    \frametitle{Divide et impera}
    \framesubtitle{Problema di programmazione}

    \centering
    \smartarttree[
        Contare il numero di parole diverse\\presenti in tutti i file del computer 
        [Cercare tutti i\\file del computer
        ]
        [Contare le parole\\diverse in un file
        ]
    ]
\end{contentframe}

\begin{contentframe}
    \frametitle{Divide et impera}
    \framesubtitle{Problema di programmazione}

    \centering
    \smartarttree[
        Contare il numero di parole diverse\\presenti in tutti i file del computer 
        [Cercare tutti i\\file del computer
            [Cercare\\nella cartella\\corrente]
            [Cercare\\nelle\\sottocartelle]
        ]
        [Contare le parole\\diverse in un file
        ]
    ]
\end{contentframe}

\begin{contentframe}
    \frametitle{Divide et impera}
    \framesubtitle{Problema di programmazione}

    \centering
    \smartarttree[
        Contare il numero di parole diverse\\presenti in tutti i file del computer 
        [Cercare tutti i\\file del computer
            [Cercare\\nella cartella\\corrente]
            [Cercare\\nelle\\sottocartelle]
        ]
        [Contare le parole\\diverse in un file
            [Leggere il\\contenuto\\del file]
            [Separare\\le\\parole]
            [Aggiungere la\\parola se non\\è già presente]
        ]
    ]
\end{contentframe}

\begin{exampleframe}
    \frametitle{Divide et impera}

    \begin{itemize}
        \item Vediamo un esempio pratico
        \bigskip
        \item Vogliamo creare un programma che faccia le seguenti azioni...
    \end{itemize}
\end{exampleframe}

\begin{exampleframe}
    \frametitle{Divide et impera}

    \begin{enumerate}
        \item Chiedere all'utente il nome di un file
        \begin{itemize}
            \item Il nome non può essere vuoto e non può contenere i caratteri `\texttt{?}', `\texttt{/}' e `\texttt{*}'
            \item Se il nome non è valido, va richiesto fino a che non sia corretto
        \end{itemize}
        \item Aprire il file e leggere il suo contenuto
        \item Contare il numero di vocali contenute nel file
        \item Chiedere all'utente il nome di un secondo file
        \begin{itemize}
            \item Come prima, se il nome non è corretto, richiederlo
        \end{itemize}
        \item Aprire il secondo file e scrivere il conteggio effettuato in precedenza
    \end{enumerate}
\end{exampleframe}

\begin{exampleframe}
    \frametitle{Divide et impera}
    \framesubtitle{Una possibile soluzione}

    \centering
    \image*[1][.7]{functions_bad.jpg}
\end{exampleframe}

\begin{exampleframe}
    \frametitle{Divide et impera}
    \framesubtitle{Osservazioni}

    \begin{itemize}
        \item Come vediamo il codice non è molto chiaro

        \bigskip
        \item Applichiamo ``divide et impera'' ed estraiamo dei sottoproblemi:
        \begin{itemize}
            \item Capire se il nome del file è corretto
            \item Ottenere il nome di un file, assicurandosi sia corretto
            \item Capire se una lettera è una vocale
            \item Contare le vocali in una stringa
        \end{itemize}
    \end{itemize}
\end{exampleframe}

\begin{exampleframe}
    \frametitle{Divide et impera}
    \framesubtitle{Creiamo delle funzioni}

    \centering
    \image*[1][.7]{functions1.jpg}
\end{exampleframe}

\begin{exampleframe}
    \frametitle{Divide et impera}
    \framesubtitle{Codice finale}

    \centering
    \image*[1][.7]{functions2.jpg}
\end{exampleframe}

\begin{exerciseframe}
    \frametitle{Esercizio finale}

    Creare un programma che faccia le seguenti azioni.
    Organizzare il codice in funzioni come spiegato.

    \begin{enumerate}
        \item Chiedere all'utente il nome di un file
        \begin{itemize}
            \item Il nome non può essere vuoto e non può contenere i caratteri `\texttt{?}', `\texttt{/}' e `\texttt{*}'
            \item Se il nome non è valido, va richiesto fino a che non sia corretto
        \end{itemize}
        \item Aprire il file e leggere il suo contenuto
        \item Contare il numero di vocali contenute nel file
        \item Chiedere all'utente il nome di un secondo file
        \begin{itemize}
            \item Come prima, se il nome non è corretto, richiederlo
        \end{itemize}
        \item Aprire il secondo file e scrivere il conteggio effettuato in precedenza
    \end{enumerate}
\end{exerciseframe}


% TODO
%   aggiungere named arguments
%       print(3, 4, sep='-', end='')
%   aggiungere *args e **kwargs