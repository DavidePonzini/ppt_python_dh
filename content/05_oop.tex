\makesectionframe{Object-Oriented Programming}[Programmazione orientata ad oggetti]

\begin{contentframe}
    \frametitle{Object-Oriented Programming}

    \begin{itemize}
        \item Tecnica di programmazione
        \begin{itemize}
            \item Stesso linguaggio
            \item Stessi programmi
            \item Diverso metodo di organizzare il codice
        \end{itemize}
    \end{itemize}
\end{contentframe}

\begin{contentframe}
    \frametitle{Object-Oriented Programming}

    \begin{itemize}
        \item Ogni ``entità'' è vista come un \textbf{oggetto}

        \bigskip
        \item Su quell'oggetto applico diverse operazioni ed ottengo un altro oggetto
        \item Ripeto fino ad arrivare al risultato

        \bigskip
        \item I programmi non sono più una sequenza di azioni, ma un insieme di oggetti che interagiscono tra di loro
    \end{itemize}
\end{contentframe}

\begin{contentframe}
    \frametitle{Perché?}
    \framesubtitle{Semplicità concettuale}

    \begin{columns}
        \col{.5}
        \textbf{Senza oggetti}
        \begin{itemize}
            \item Devo tenere a mente l'intera situazione in ogni momento
        \end{itemize}
        
        \col{.5}
        \textbf{Con oggetti}
        \begin{itemize}
            \item Tengo a mente solo una parte del problema alla volta
        \end{itemize}
    \end{columns}
\end{contentframe}

\begin{contentframe}
    \frametitle{Perché?}
    \framesubtitle{Organizzazione codice}

    \begin{columns}
        \col{.5}
        \textbf{Senza oggetti}
        \begin{itemize}
            \item Ho molte funzioni diverse nello stesso blocco di codice
        \end{itemize}
        
        \col{.5}
        \textbf{Con oggetti}
        \begin{itemize}
            \item In un certo blocco di codice ho solo funzioni relative ad un singolo concetto
        \end{itemize}
    \end{columns}
\end{contentframe}

\begin{contentframe}
    \frametitle{Perché?}
    \framesubtitle{Semplicità programmazione}

    \begin{columns}
        \col{.5}
        \textbf{Senza oggetti}
        \begin{itemize}
            \item Ho a disposizione in ogni momento tutte le funzioni esistenti
        \end{itemize}
        
        \col{.5}
        \textbf{Con oggetti}
        \begin{itemize}
            \item Ho a disposizione solo le funzione rilevanti ai dati che sto usando
        \end{itemize}
    \end{columns}
\end{contentframe}

\makesectionframe{Classi}

\begin{contentframe}
    \frametitle{Classi}

    \begin{columns}
        \col{.5}
        \begin{itemize}
            \item Usate per rapprentare gli oggetti
            \begin{itemize}
                \item Secondo la OOP, i programmi sono classi che interagiscono tra di loro
            \end{itemize}

            \bigskip
            \item Classe
            \begin{itemize}
                \item Insieme di dati (e funzioni)
                \item Relativa ad un singolo concetto
            \end{itemize}
        \end{itemize}
        
        \col{.5}
        \centering
        \image{class.jpg}
        {
            \footnotesize
            \textit{Questo è un esempio, la sintassi reale è leggermente diversa}
        }
    \end{columns}
\end{contentframe}


\begin{contentframe}
    \frametitle{Metodi}

    \begin{columns}
        \col{.5}
        \begin{itemize}
            \item Funzioni contenute dentro ad una classe che operano sui dati di quella classe

            \bigskip
            \item Per accedere ai dati della classe, devo usare \texttt{self}
            \begin{itemize}
                \item Salvo rare eccezioni, \texttt{self} è il primo argomento di ogni metodo
            \end{itemize}
        \end{itemize}
        
        \col{.5}
        \centering
        \image{class_method.jpg}
    \end{columns}
\end{contentframe}

\begin{contentframe}
    \frametitle{Metodi}

    \begin{columns}
        \col{.5}
        \begin{itemize}
            \item Nel programma, quando ho una persona, ho a dispozione anche un metodo per controllare se è minorenne

            \bigskip
            \item Per accedere agli elementi contenuti in un oggetto, uso il `\texttt{.}' \textit{(dot-notation)}
        \end{itemize}
        
        \col{.5}
        \centering
        \image{class_method_usage.jpg}
    \end{columns}
\end{contentframe}

\begin{contentframe}
    \frametitle{Costruttore}

    \begin{columns}
        \col{.5}
        \begin{itemize}
            \item Metodo per inizializzare una classe
            \item Posso anche passare dei valori

            \bigskip
            \item La funzione si chiama \texttt{\_\_init\_\_}
        \end{itemize}
        
        \col{.5}
        \centering
        \image{class_init.jpg}
    \end{columns}
\end{contentframe}

\begin{contentframe}
    \frametitle{Costruttore}

    \begin{columns}
        \col{.5}
        \begin{itemize}
            \item Nel programma principale, passo i valori quando creo l'oggetto
        \end{itemize}
        
        \col{.5}
        \centering
        \image{class_init_usage.jpg}
    \end{columns}
\end{contentframe}

\begin{exerciseframe}
    \frametitle{Esercizio (1/2)}

    \begin{itemize}
        \item Creare la classe \texttt{Item}, che contiene:
        \begin{itemize}
            \item \texttt{name} (Nome dell'articolo)
            \item \texttt{price} (Prezzo dell'articolo)
        \end{itemize}

        \bigskip
        \item Creare la classe \texttt{Person}, che contiene:
        \begin{itemize}
            \item \texttt{name} (Nome della persona)
            \item \texttt{money} (Soldi a disposizione della persona)
            \item \texttt{items} (Oggetti posseduti dalla persona. Al momento della creazione: nessuno)
            \item \texttt{buy(self, item)} (Metodo che acquista un oggetto, se la persona se lo può permettere)
        \end{itemize}
    \end{itemize}
\end{exerciseframe}

\begin{exerciseframe}
    \frametitle{Esercizio (2/2)}

    \begin{itemize}
        \item Nel programma principale, eseguire le seguenti azioni:
        \begin{enumerate}
            \item Creare la persona `Mario', che ha a disposizione 25€
            \item Creare una mela da 7€
            \item Creare una pera da 5€
            \item Creare un limone da 25€
            \item Creare un arancio da 5€
            \item Far acquistare a Mario una mela, due pere, un limone e tre arance
            \item Stampare quanti soldi ha Mario
            \item Stampare gli articoli che Mario possiede
            \pause
            \bigskip
            \item Provare ad eseguire queste azioni con più persone
        \end{enumerate}
    \end{itemize}
\end{exerciseframe}