\makesectionframe{Librerie}

\begin{contentframe}
    \frametitle{Librerie}

    \begin{itemize}
        \item Insieme di funzioni già pronte scritte da altre persone
        \bigskip
        \item Esistono numerose librerie
        \item Ciascuna si occupa (generalmente) di una sola cosa
        \bigskip
        \item Per usarle nel mio codice uso il comando \boxed{\texttt{import} \textit{libreria}}
        \begin{itemize}
            \item Una volta importate, per usare le funzionalità mi basta scrivere:\\
            \texttt{libreria.funzione}
        \end{itemize}
    \end{itemize}
\end{contentframe}

\begin{exampleframe}
    \frametitle{Librerie}
    \framesubtitle{Esempio}

    \centering
    \only<1>{
        \image*[1][.7]{import.jpg}
    }
    \only<2>{
        \image*[1][.7]{import2.jpg}
    }
\end{exampleframe}

\begin{contentframe}
    \frametitle{Librerie}
    \framesubtitle{random}

    \begin{columns}
        \col{.5}
        \begin{itemize}
            \item Permette di generare numeri casuali

            \bigskip
            \item Le sequenze sono \textbf{pseudocasuali}
            \begin{itemize}
                \item Sembrano casuali ma non lo sono
                \item Dipendono dal numero (``seed'') con cui sono inizializzate
            \end{itemize}
        \end{itemize}
        
        \col{.5}
        \centering
        \image{random.jpg}
    \end{columns}
\end{contentframe}

\begin{exerciseframe}
    \frametitle{Esercizio guidato}
    \framesubtitle{Inizializzazione sequenze random}

    \begin{columns}
        \col{.5}
        \begin{enumerate}
            \item Eseguire più volte il seguente programma
            \item Provare a cambiare il valore del seed ed osservare le differenze
        \end{enumerate}
        
        \col{.5}
        \centering
        \image{random_seed.jpg}
    \end{columns}

    \pause
    \bigskip
    \begin{itemize}
        \item Quale seed mi può permettere di avere sequenze diverse ogni volta che eseguo il programma?
    \end{itemize}
\end{exerciseframe}

\begin{exampleframe}
    \frametitle{Random}

    \begin{itemize}
        \item Lo stesso seed porta sempre alla stessa sequenza di numeri casuali

        \bigskip
        \item Per avere ogni volta una sequenza diversa devo inizializzare seed con l'orario corrente
        \item In Python, questa azione è fatta in automatico quando si importa la libreria
    \end{itemize}
\end{exampleframe}

\begin{exerciseframe}
    \frametitle{Esercizi}
    \framesubtitle{Libreria random}

    \begin{enumerate}
        \item Stampare un numero a caso. Ogni volta che il programma viene eseguito deve essere stampato un numero diverso

        \pause
        \bigskip
        \item Simulare il lancio di una moneta. Il programma dovrà stampare ``Testa'' o ``Croce''

        \pause
        \bigskip
        \item Scegliere un numero a caso tra 0 e 9. L'utente ha 3 tentativi per indovinare il numero. Se sbaglia, dire se il numero da indovinare è maggiore o minore di quello inserito
    \end{enumerate}
\end{exerciseframe}

\begin{contentframe}
    \frametitle{Librerie}
    \framesubtitle{time}

    \begin{columns}
        \col{.5}
        \begin{itemize}
            \item Operazioni relative al tempo
            \bigskip
            \item \texttt{sleep} permette di bloccare l'esecuzione del programma per un certo numero di secondi
        \end{itemize}
        
        \col{.5}
        \centering
        \image{time.jpg}
    \end{columns}
\end{contentframe}

\begin{contentframe}
    \frametitle{Librerie}
    \framesubtitle{datetime}

    \begin{columns}
        \col{.5}
        \begin{itemize}
            \item Operazioni su date e ore
            \bigskip
            \item Contiene:
            \begin{itemize}
                \item \texttt{datetime.date}
                \item \texttt{datetime.time}
                \item \texttt{datetime.datetime}
            \end{itemize}
            \bigskip
            \item \texttt{datetime.datetime.now} ritorna il timestamp corrente 
        \end{itemize}
        
        \col{.5}
        \centering
        \image{datetime.jpg}
    \end{columns}
\end{contentframe}

\begin{exerciseframe}
    \frametitle{Esercizi (1)}
    \framesubtitle{Libreria datetime}

    \begin{enumerate}
        \item Stampare la data corrente
    
        \pause
        \bigskip
        \item Creare un programma che dica ``Aspetta un secondo...'' e dopo una pausa di un secondo stampi ``Fatto!''
    \end{enumerate}
\end{exerciseframe}

\begin{exerciseframe}
    \frametitle{Esercizi (2)}
    \framesubtitle{Libreria datetime}

    Eseguire il seguente programma più volte. Perché i tempi cambiano?

    \medskip
    \image[1][.5]{time_sleep.jpg}
\end{exerciseframe}

\makesectionframe{Gestione librerie}

\begin{contentframe}
    \frametitle{Gestione librerie}

    \begin{itemize}
        \item Le librerie che abbiamo usato finora sono automaticamente installate in Python
        \item Possiamo anche installare nuove librerie

        \bigskip
        \item Usiamo il modulo pip di Python
        \begin{itemize}
            \item \texttt{python -m pip install nuovalibreria}
        \end{itemize}
    \end{itemize}
\end{contentframe}

\begin{contentframe}
    \frametitle{Ambienti virtuali}

    \begin{itemize}
        \item \texttt{pip} installa in maniera globale
        \item Spesso non è desiderabile
        \begin{itemize}
            \item Esempio: abbiamo due progetti che richiedono due diverse versioni della stessa libreria
        \end{itemize}

        \bigskip
        \item Soluzione: si usano gli ambienti virtuali
    \end{itemize}
\end{contentframe}

\begin{contentframe}
    \frametitle{Ambienti virtuali}

    \begin{itemize}
        \item L'ambiente virtuale è di fatto una cartella
        \item Contiene le librerie relative ad un solo progetto
        \item Si può cancellare e ricreare senza alcun problema

        \item Consiglio: scrivere tutte le librerie che scaricate in un file chiamato \texttt{requirements.txt}

        \bigskip
        \item Per crearlo si usa il comando \texttt{python -m venv venv}
        \begin{itemize}
            \item Il primo \texttt{venv} è il modulo, il secondo è il nome della cartella che creeremo
        \end{itemize}
        \item Per usarlo dobbiamo prima attivarlo
        \begin{itemize}
            \item Windows: \texttt{./venv/Scripts/activate}
            \item Linux, Mac: \texttt{source ./venv/bin/activate}
        \end{itemize}
    \end{itemize}
\end{contentframe}

\begin{exampleframe}
    \frametitle{Ambienti virtuali}
    \framesubtitle{Guida alla creazione (1)}

    \begin{enumerate}
        \item Scrivere tutte le librerie che vi servono in un file chiamato \texttt{requirements.txt}
        \begin{itemize}
            \item Scrivere una sola libreria per riga
        \end{itemize}

        \bigskip
        \item Creare l'ambiente virtuale tramite il comando \boxed{\texttt{python -m venv venv}}
        \begin{itemize}
            \item Il primo \texttt{venv} è il modulo, il secondo è il nome della cartella che creeremo
        \end{itemize}
        
        \bigskip
        \item Attivare l'ambiente virtuale
        \begin{itemize}
            \item Windows: \boxed{\texttt{./venv/Scripts/activate}}
            \item Linux, Mac: \boxed{\texttt{source ./venv/bin/activate}}
        \end{itemize}
    \end{enumerate}
\end{exampleframe}

\begin{exampleframe}
    \frametitle{Ambienti virtuali}
    \framesubtitle{Guida alla creazione (2)}

    \begin{enumerate}
        \item Attivare l'ambiente virtuale
        \begin{itemize}
            \item Windows: \boxed{\texttt{./venv/Scripts/activate}}
            \item Linux, Mac: \boxed{\texttt{source ./venv/bin/activate}}
        \end{itemize}

        \bigskip
        \item Installare le librerie indicate in \texttt{requirements.txt}
        \begin{itemize}
            \item \boxed{\texttt{python -m pip install -r requirements.txt}}
        \end{itemize}

        \bigskip
        \item Usare Python normalmente!
    \end{enumerate}
\end{exampleframe}

\begin{contentframe}
    \frametitle{Librerie}
    \framesubtitle{argparse}

    \begin{columns}
        \col{.5}
        \begin{itemize}
            \item 
        \end{itemize}
        
        \col{.5}
        \centering
        % \image{datetime.jpg}
    \end{columns}
\end{contentframe}

\begin{contentframe}
    \frametitle{Librerie}
    \framesubtitle{pandas}

    \begin{columns}
        \col{.5}
        \begin{itemize}
            \item 
        \end{itemize}
        
        \col{.5}
        \centering
        % \image{datetime.jpg}
    \end{columns}
\end{contentframe}

