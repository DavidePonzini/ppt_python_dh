\makesectionframe{Utilizzo del terminale}
\begin{contentframe}
    \frametitle{Il terminale}

    \begin{columns}
        \col{.5}
        \begin{itemize}
            \item I termini \textit{``terminale''}, \textit{``shell''}, \textit{``prompt dei comandi''} e \textit{``command line''} si riferiscono alla stessa cosa

            \bigskip
            \item Useremo il terminale spessissimo in questo corso, è molto importante imparare ad usarlo bene fin da subito
        \end{itemize}

        \col{.5}
        \centering
        \image[1][.7]{shell.jpg}
    \end{columns}
\end{contentframe}

\begin{contentframe}
    \frametitle{Utilizzo del terminale}

    \begin{itemize}
        \item Interfaccia grafica minimale per eseguire comandi e visualizzarne l'output
        \begin{itemize}
            \item Supporta solo formati testuali
        \end{itemize}

        \bigskip
        \item I comandi sono eseguiti quando premo \textbf{Invio}
        \item Eventuali parametri sono separati tramite il carattere \textbf{Spazio}
    \end{itemize}
\end{contentframe}

\begin{contentframe}
    \frametitle{Utilizzo del terminale}
    \framesubtitle{Cambiare cartelle}

    \begin{itemize}
        \item La riga corrente del prompt mi indica sempre in che cartella mi trovo
        \begin{itemize}
            \item Si può personalizzare per indicare anche altre informazioni, ma è fuori dallo scope di questo corso 
        \end{itemize}
        
        \bigskip
        \item Per cambiare cartella si usa il comando \boxed{\texttt{cd}} \textit{(Change Directory)}
        \begin{itemize}
            \item \boxed{\texttt{cd NomeCartella}}
            
        \end{itemize}
    \end{itemize}
\end{contentframe}

\begin{exampleframe}
    \frametitle{Utilizzo del terminale}
    \framesubtitle{Cambiare cartelle -- errori comuni}

    \centering
    \boxed{\texttt{cd NomeCartella}}
    \bigskip
    
    \begin{itemize}
        \item Lo spazio è importante: separa il nome del comando dal nome del parametro
        \begin{itemize}
            \item \texttt{cdCartella} è un altro comando (che non esiste)
        \end{itemize}
        \bigskip
        
        \item Se la nostra cartella ha degli spazi bisogna racchiuderla tra virgolette
        \begin{itemize}
            \item \texttt{cd "La mia cartella"} oppure \texttt{cd \textquotesingle{}La mia cartella\textquotesingle{}} 
        \end{itemize}
        
        \bigskip
        \item I caratteri \texttt{"} e \texttt{\textquotesingle{}} sono diversi da ``, '', ` e '
        \begin{itemize}
            \item Particolarmente importante quando si fa copia-incolla
        \end{itemize}
    \end{itemize}
\end{exampleframe}

\begin{exampleframe}
    \frametitle{Utilizzo del terminale}
    \framesubtitle{Cambiare cartelle -- nomi speciali}

    
    \begin{itemize}
        \item Alcune cartelle hanno dei nomi speciali:
        \begin{itemize}
            \item \boxed{\texttt{cd .}}~: cartella corrente
            \item \boxed{\texttt{cd ..}}~: livello superiore
            \item \boxed{\texttt{cd \tilde}}~: cartella home
            \item \boxed{\texttt{cd /}}~: root directory \textit{(cartella più ``in alto'')}
        \end{itemize}
        \bigskip

        \item Su Windows, per cambiare partizione \textit{(es. da \texttt{C:} a \texttt{D:})} non si usa \texttt{cd} ma si scrive solo la partizione che ci interessa
        \begin{itemize}
            \item \boxed{\texttt{d:}}
        \end{itemize}
    \end{itemize}
\end{exampleframe}

\begin{contentframe}
    \frametitle{Utilizzo del terminale}
    \framesubtitle{Mostrare contenuto cartelle}

    \begin{itemize}
        \item Per visualizzare il contenuto si usa il comando:
        \begin{itemize}
            \item \boxed{\texttt{dir}} \textit{(DIRectory)} -- Windows
            \item \boxed{\texttt{ls}} \textit{(LiSt)} -- Linux e Mac
        \end{itemize}
    \end{itemize}
\end{contentframe}

\begin{contentframe}
    \frametitle{Utilizzo del terminale}
    \framesubtitle{Suggerimenti}

    \begin{itemize}
        \item Il tasto \texttt{Tab} autocompleta spesso comandi e nomi di file/cartelle

        \bigskip
        \item Le frecce su e giù permettono di ciclare tra gli ultimi comandi eseguiti
        \item Le frecce sinistra e destra permettono di spostarsi nel testo del comando corrente
        
        \bigskip
        \item Il mouse non serve a niente

        \bigskip
        \item \texttt{Ctrl+C} interrompe forzatamente l'esecuzione del comando corrente
        \begin{itemize}
            \item Non fa copia!
        \end{itemize}
    \end{itemize}
\end{contentframe}

\begin{exerciseframe}
    \frametitle{Utilizzo del terminale}
    \frametitle{Esercizio}

    \begin{enumerate}
        \item Creare una cartella sul Desktop \textit{(o dove preferite)}
        \item Creare un file in quella cartella
        \item Aprire il terminale
        \item Navigare fino alla cartella creata
        \item Mostrarne il contenuto
    \end{enumerate}
\end{exerciseframe}

\makesectionframe{Python}

\begin{exampleframe}
    \frametitle{Curiosità}

    \begin{itemize}
        \item Python non ha nulla a che fare con i rettili
        \item Prende il nome dallo show ``Monty Python's Flying Circus''
    \end{itemize}

    \bigskip
    \begin{columns}
        \col{.4}
        \centering
        \image{pythons1.jpg}
        
        \col{.4}
        \centering
        \image{pythons2.jpg}
    \end{columns}
\end{exampleframe}

\begin{contentframe}
    \frametitle{Python}

    \begin{itemize}
        \item Linguaggio di programmazione ad alto livello
        \begin{itemize}
            \item Sequenza di istruzioni che il computer esegue
            \item Si può fare qualunque cosa
            \item Alto livello: non mi devo preoccupare dei dettagli più fini
        \end{itemize}
        
        \bigskip
        \item Linguaggio interpretato
        \begin{itemize}
            \item Il codice non ha bisogno di essere compilato
        \end{itemize}

        \bigskip
        \item Esistono tanti moduli con funzionalità già fatte
        \begin{itemize}
            \item Permettono di risparmiare tempo
        \end{itemize}
    \end{itemize}
\end{contentframe}

\begin{exampleframe}
    \frametitle{Linguaggi a basso ed alto livello}

    \begin{itemize}
        \item Basso livello
        \begin{itemize}
            \item Devo indicare in dettaglio come eseguire qualsiasi azione
            \begin{itemize}
                \item \textit{Metti il valore 7 sul registro 3 della CPU, poi trasferisci il contenuto del registro 3 sulla cella 4728 della RAM} 
            \end{itemize}
            \item Esempio: Assembly
        \end{itemize}

        \bigskip
        \item Alto livello
        \begin{itemize}
            \item Molto più astratto, indico cosa fare e non come farlo
            \begin{itemize}
                \item \textit{\texttt{x = 7}}
            \end{itemize}
            \item Esempi: Python, C++, Java
        \end{itemize}
    \end{itemize}
\end{exampleframe}

\begin{exampleframe}
    \frametitle{Linguaggi compilati ed interpretati}

    \begin{itemize}
        \item Linguaggio compilato
        \begin{itemize}
            \item Tramite un programma apposito (compilatore), il codice viene convertito in istruzioni macchina
            \item Il programma risultante (.exe) si può poi eseguire direttamente senza più bisogno del compilatore
            \item Esempio: C, C++
        \end{itemize}

        \bigskip
        \item Linguaggio interpretato
        \begin{itemize}
            \item Le istruzioni sono lette ed eseguite da un programma apposito (interprete)
            \item Senza interprete, il codice non si può eseguire
            \item Esempi: Python, Java, Javascript
        \end{itemize}
    \end{itemize}
\end{exampleframe}

\begin{contentframe}
    \frametitle{Modalità interattiva VS modalità script}

    \begin{itemize}
        \item L'interprete Python può essere eseguito in due modalità

        \bigskip
        \item Modalità interattiva
        \begin{itemize}
            \item Si scrivono i comandi direttamente nel terminale
            \item Sono eseguiti non appena scritti
            \item Stampa tutti i valori in output
        \end{itemize}

        \bigskip
        \item Modalità script
        \begin{itemize}
            \item Si scrivono i comandi in un file
            \item Si esegue l'intero file
            \item Stampa solo se si usa la funzione \texttt{print()}
        \end{itemize}
    \end{itemize}
\end{contentframe}

\begin{contentframe}
    \frametitle{Modalità interattiva VS modalità script}

    \begin{itemize}
        \item La modalità interattiva è consigliata per:
        \begin{itemize}
            \item Vedere velocemente l'output di un certo comando
            \item Fare prove o esperimenti
        \end{itemize}

        \bigskip
        \item La modalità script è consigliata per:
        \begin{itemize}
            \item Eseguire più volte lo stesso codice senza doverlo riscrivere ogni volta
        \end{itemize}
    \end{itemize}
\end{contentframe}


\begin{exerciseframe}
    \frametitle{Esercizio guidato (1)}
    \framesubtitle{Modalità interattiva}

    \begin{enumerate}
        \item Aprire una shell ed eseguire il comando \boxed{\texttt{python}}\footnote[frame]{Se il computer non trova il comando, provate con \texttt{python3}} o il comando \boxed{\texttt{ipython}}\footnote[frame]{Fa le stesse cose di \texttt{python}, ma è esteticamente più carino}
        \item Scrivere ciascuno dei seguenti comandi e premere Invio:
            \begin{itemize}
                \item \texttt{1 + 1}
                \pause
                \item \texttt{2 > 2}
                \pause
                \item \texttt{2 + 1 > 2}
                \item \texttt{4 + 6 * 2 == 10}
                \item \texttt{(4 + 6) * 2 == 10}
                \item \texttt{5 - 5 == 3 or 1 < 2}
                \item \texttt{5 / 2}
            \end{itemize}
    \end{enumerate}
\end{exerciseframe}

\begin{exerciseframe}
    \frametitle{Esercizio guidato (2)}
    \framesubtitle{Modalità interattiva}

    \begin{enumerate}
        \item Aprire una shell ed eseguire il comando \boxed{\texttt{python}}\footnote[frame]{Se il computer non trova il comando, provate con \texttt{python3}} o il comando \boxed{\texttt{ipython}}\footnote[frame]{Fa le stesse cose di \texttt{python}, ma è esteticamente più carino}
        \item Scrivere ciascuno dei seguenti comandi e premere Invio:
            \begin{itemize}
                \item \texttt{\textquotesingle{}Nome\textquotesingle{} == \textquotesingle{}Davide\textquotesingle{}}
                \item \texttt{len(\textquotesingle{}Python\textquotesingle{})}
                \item \texttt{len(\textquotesingle{}Ciao come stai?\textquotesingle{}) > len(\textquotesingle{}Ciaocomestai\textquotesingle{})}
            \end{itemize}
    \end{enumerate}
\end{exerciseframe}


\begin{exerciseframe}
    \frametitle{Esercizio guidato}
    \framesubtitle{Modalità script}

    \begin{enumerate}
        \item Creare un file chiamato \texttt{prova1.py}
        \item Scrivere nel file \boxed{\texttt{print(\textquotesingle{}Hello, world!\textquotesingle{})}}
        \item Salvare il file
        \item Scrivere nella shell \boxed{\texttt{python prova1.py}}
        \begin{itemize}
            \item Assicuratevi di essere nella cartella corretta!
        \end{itemize}

        \pause
        \bigskip
        \item Modificare il file in un modo a vostra scelta
        \item Eseguire nuovamente lo script
        \begin{itemize}
            \item Se l'output non è cambiato, è perché non avete salvato il file
        \end{itemize}
    \end{enumerate}
\end{exerciseframe}

\makesectionframe{Concetti base}

\begin{contentframe}
    \frametitle{Funzioni}

    \begin{itemize}
        \item Sequenza di azioni che fa una cosa

        \bigskip
        \item Esempio: \fbox{\texttt{print(...)}} stampa dati su schermo

        \bigskip
        \item Sintassi: \texttt{nomedellafunzione(argomento1, argomento2, ...)}
        \begin{itemize}
            \item Le parentesi sono obbligatorie SEMPRE
            \item Gli argomenti sono i valori su cui voglio eseguire le operazioni
        \end{itemize}
    \end{itemize}
\end{contentframe}

\begin{exerciseframe}
    \frametitle{Esercizio guidato}
    \framesubtitle{Funzioni - \texttt{print()}}

    Cosa stampano le seguenti funzioni?
    \begin{itemize}
        \item \texttt{print(1)}\pause
        \item \texttt{print(5)}\pause
        \item \texttt{print(3, 5)}\pause
        \item \texttt{print(1, 2, 3, 4, 5)}\pause
        \item \texttt{print()}\pause
        \item \texttt{print}\pause{} -- Attenzione: non viene eseguita!
    \end{itemize}
\end{exerciseframe}

\begin{contentframe}
    \frametitle{Variabili}

    \begin{columns}
        \col{.5}
        \begin{itemize}
            \item Permettono di memorizzare dati o risultati

            \bigskip
            \item Possono avere (quasi) qualunque nome
            
            \bigskip
            \item Hanno diversi tipi, in base a che dati vogliamo salvare
            \begin{itemize}
                \item Python li capisce in automatico, ma è importante saperli riconoscere
            \end{itemize}
        \end{itemize}
        
        \col{.5}
        \centering
        \image{variables.jpg}
    \end{columns}
\end{contentframe}

\begin{contentframe}
    \frametitle{Variabili}
    \framesubtitle{Numeri}

    \begin{columns}
        \col{.5}
        \centering
        {\Huge\faCalculator}\\
        \bigskip
        \texttt{int}\\
        \textbf{Numero senza virgola}\\
        \bigskip
        1\\124\\-336

        \col{.5}
        \centering
        {\Huge\faCalculator}\\
        \bigskip
        \texttt{float}\\
        \textbf{Numero con virgola}\\
        \bigskip
        -3.14\\5.0\\\bigskip
    \end{columns}
\end{contentframe}

\begin{contentframe}
    \frametitle{Variabili}
    \framesubtitle{Testo}

    \begin{columns}
        \col{1}
        \centering
        {\Huge\faPen}\\
        \bigskip
        \texttt{str}\\
        \textbf{Stringa}\\
        \bigskip
        \texttt{"}ciao\texttt{"}\\
        \texttt{"}hello\texttt{"}\\
        \texttt{"}come stai?\texttt{"}
    \end{columns}
\end{contentframe}

\begin{contentframe}
    \frametitle{Variabili}
    \framesubtitle{Logica}

    \begin{columns}
        \col{1}
        \centering
        {\Huge\faCheck}\\
        \bigskip
        \texttt{bool}\\
        \textbf{Booleano}\\
        \bigskip
        True\\False
    \end{columns}
\end{contentframe}

\begin{contentframe}
    \frametitle{Variabili}
    \framesubtitle{Strutture dati base}

    \begin{columns}
        \col{.3}
        \centering
        {\Huge\faListOl}\\
        \bigskip
        \texttt{list}\\
        \textbf{Lista}\\
        (sequenza ordinata di elementi)\\
        \bigskip
        [1, 2, 2, 3]\\
        
        [\texttt{"}ciao\texttt{"}, \texttt{"}come\texttt{"}, \texttt{"}stai\texttt{"}]

        \col{.3}
        \centering
        {\Huge\faBars}\\
        \bigskip
        \texttt{set}\\
        \textbf{Set}\\
        (sequenza senza ripetizioni e senza ordine)\\
        \bigskip
        \{1, 5, 9\}\\
        \{True, False\}

        \col{.3}
        \centering
        {\Huge\faThList}\\
        \bigskip
        \texttt{dict}\\
        \textbf{Mappa/Dizionario}\\
        (coppie chiave-valore)\\
        \bigskip
        \bigskip
        \{\textquotesingle{}Paolo\textquotesingle{}: 8, \textquotesingle{}Alice\textquotesingle{}: 9\}
        \bigskip
    \end{columns}
\end{contentframe}

\begin{contentframe}
    \frametitle{Variabili}
    \framesubtitle{Altro}

    \begin{columns}
        \col{.5}
        \centering
        {\Huge\faSitemap}\\
        \bigskip
        \texttt{class}\\
        \textbf{Classi}\\
        \bigskip
        Insieme di dati di tipologie diverse

        \col{.5}
        \centering
        {\Huge\faTimes}\\
        \bigskip
        \texttt{None}\\
        \textbf{Nessun valore}\\
        \bigskip
        \bigskip
    \end{columns}
\end{contentframe}

\begin{contentframe}
    \frametitle{Variabili}
    \framesubtitle{Stringhe}

    \begin{itemize}
        \item Le stringhe si possono delimitare con i caratteri \texttt{"} o \textquotesingle{}
        \begin{itemize}
            \item \texttt{"Stringa"} oppure \texttt{\textquotesingle{}Stringa\textquotesingle{}}
        \end{itemize}

        \bigskip
        \item Per stampare alcuni caratteri speciali, devo usare una sequenza di escape
        \begin{itemize}
            \item \texttt{$\backslash$n} $\rightarrow$ A capo
            \item \texttt{$\backslash$t} $\rightarrow$ Tabulazione
            \item \texttt{$\backslash$r} $\rightarrow$ Carriage Return (riporta a inizio riga)
            \item \texttt{$\backslash\backslash$} $\rightarrow$ Carattere \texttt{$\backslash$}
            \item \texttt{$\backslash$\textquotesingle{}} o \texttt{$\backslash$"} $\rightarrow$ Caratteri \texttt{\textquotesingle{}} o \texttt{"} (da usare solo quando necessario)
        \end{itemize}
    \end{itemize}
\end{contentframe}

\begin{exerciseframe}
    \frametitle{Esercizio}
    \framesubtitle{Sequenze di escape}

    \begin{itemize}
        \item Stampare le seguenti stringhe
        \begin{itemize}
            \item \textit{Queste sono\\due righe}
            \smallskip
            \item \textit{Oggi c'è il sole!}
            \smallskip
            \item \textit{J. K. Rowling ha scritto ``Harry Potter''.}
            \smallskip
            \item \textit{L'autore de ``Il Signore degli Anelli'' è J. R. R. Tolkien.}
        \end{itemize}
    \end{itemize}
\end{exerciseframe}


\begin{contentframe}
    \frametitle{Variabili}
    \framesubtitle{Casting}

    \begin{itemize}
        \item Permette di convertire una variabile da un tipo ad un altro

        \bigskip
        \item Si usano (generalmente) funzioni omonime al tipo che vogliamo ottenere
        \begin{itemize}
            \item \texttt{int(3.5)} $\rightarrow$ \texttt{3}
            \item \texttt{int(\textquotesingle{}11\textquotesingle{})} $\rightarrow$ \texttt{11}
            \item \texttt{set([1, 2, 2, 3])} $\rightarrow$ \texttt{\{1, 2, 3\}}
        \end{itemize}

        \bigskip
        \item Per vedere di che tipo è un dato si può usare la funzione \boxed{\texttt{type(}\textit{valore}\texttt{)}}
        
    \end{itemize}
\end{contentframe}


\begin{exerciseframe}
    \frametitle{Esercizio guidato (1)}
    \framesubtitle{Variabili}

    \begin{enumerate}
        \item Salvare ciascun valore in una variabile.\\
    Stampare ciascun valore affiancato dal proprio tipo.
        \begin{itemize}
            \item 7\pause
            \item 5.5\pause
            \item ``Ciao come stai?''\pause
            \item True\pause
            \item{} [0, 1, 1, 2, 3, 5, 8, 13]\pause
            \item \{0, 1, 1, 2, 3, 5, 8, 13\}\pause
            \item  \{\textquotesingle{}Alice\textquotesingle{}: True, \textquotesingle{}Bob\textquotesingle{}: False\}
        \end{itemize}
    \end{enumerate}
\end{exerciseframe}

\begin{exerciseframe}
    \frametitle{Esercizio guidato (2)}
    \framesubtitle{Variabili e \texttt{print()}}

    \begin{enumerate}
        \setcounter{enumi}{1}
        \item Salvare i seguenti numeri in due variabili
        \begin{itemize}
            \item $x = 3$
            \item $y = 5$
        \end{itemize}

        \bigskip
        \item Stampare ``x vale 3 - y vale 5'' nei seguenti modi proposti
        \begin{itemize}
            \item \texttt{print(\textquotesingle{}x vale x - y vale y\textquotesingle{})}
            \pause
            \item \texttt{print(\textquotesingle{}x vale\textquotesingle{}, x, \textquotesingle{}- y vale\textquotesingle{}, y)}
            \pause
            \item \texttt{print(\textquotesingle{}x vale \textquotesingle{} + str(x) + \textquotesingle{} - y vale \textquotesingle{} + str(y))}
            \pause
            \item \texttt{print(\textquotesingle{}x vale \{\} - y vale \{\}\textquotesingle{}.format(x, y))}
            \pause
            \item \texttt{print(\textquotesingle{}x vale \{primo\} - y vale \{secondo\}\textquotesingle{}.format(secondo=y, primo=x))}
            \pause
            \item \texttt{print(f\textquotesingle{}x vale \{x\} - y vale \{y\}\textquotesingle{})}
        \end{itemize}
    \end{enumerate}
\end{exerciseframe}

\begin{contentframe}
    \frametitle{Operazioni matematiche}

    \centering
    \begin{tabular}{c|c|c|c}
        Operazione                      & Operatore     & Esempio           & Risultato             \\
        \midrule
        \midrule
        Addizione                       & \texttt{+}    & \texttt{7 + 3}    & \texttt{10}           \\
        Sottrazione                     & \texttt{-}    & \texttt{7 - 3}    & \texttt{4}            \\
        Moltiplicazione                 & \texttt{*}    & \texttt{7 * 3}    & \texttt{21}           \\
        Divisione                       & \texttt{/}    & \texttt{7 / 3}    & \texttt{2.333...35}   \\
        \midrule
        Elevamento a potenza            & \texttt{**}   & \texttt{7 ** 3}   & \texttt{343}          \\
        Divisione senza resto           & \texttt{//}   & \texttt{7 // 3}   & \texttt{2}            \\
        Modulo (resto della divisione)  & \texttt{\%}   & \texttt{7 \% 3}   & \texttt{1}            \\
    \end{tabular}
\end{contentframe}

\begin{contentframe}
    \frametitle{Operazioni logiche}

    \centering
    \begin{tabular}{c|c|c|c}
        Operazione                      & Operatore     & Esempio                   & Risultato         \\
        \midrule
        \midrule
        Maggiore                        & \texttt{>}    & \texttt{7 > 3}            & \texttt{True}     \\
        Maggiore o uguale               & \texttt{>=}   & \texttt{7 >= 3}           & \texttt{True}     \\
        Minore                          & \texttt{<}    & \texttt{7 < 3}            & \texttt{False}    \\
        Minore o uguale                 & \texttt{<=}   & \texttt{7 <= 3}           & \texttt{False}    \\
        Uguale                          & \texttt{==}   & \texttt{7 == 3}           & \texttt{False}    \\
        Diverso                         & \texttt{!=}   & \texttt{7 != 3}           & \texttt{True}     \\
        \midrule
        Negazione                       & \texttt{not}  & \texttt{not True}         & \texttt{False}    \\
        Congiunzione                    & \texttt{and}  & \texttt{True and False}   & \texttt{False}    \\
        Disgiunzione                    & \texttt{or}   & \texttt{True or False}    & \texttt{True}     \\
    \end{tabular}
\end{contentframe}

\begin{exerciseframe}
    \frametitle{Esercizio}
    \framesubtitle{Operazioni}

    Cosa stampa questo programma?

    \bigskip
    \image[.7]{exercise_logic.jpg}
\end{exerciseframe}

\begin{contentframe}
    \frametitle{Input}

    \begin{columns}
        \col{.5}
        \begin{itemize}
            \item La funzione \fbox{\texttt{input()}} permette di leggere un valore indicato dall'utente
    
            \bigskip
            \item Come argomento opzionale posso indicare la domanda che viene chiesta prima di leggere
        \end{itemize}
        
        \col{.5}
        \centering
        \image{input.jpg}
    \end{columns}

    \bigskip
    \begin{itemize}
        \item Attenzione: il valore ritornato è sempre di tipo \texttt{str}
        \item Per leggere un numero devo convertirlo in \texttt{int}
        \begin{itemize}
            \item \fbox{\texttt{x = int(input())}}
        \end{itemize}
    \end{itemize}
\end{contentframe}

\begin{exerciseframe}
    \frametitle{Esercizi (1)}
    \framesubtitle{Input}

    \begin{enumerate}
        \item Chiedere all'utente un numero e stamparlo

        \bigskip
        \item Chiedere all'utente un numero e stamparne il doppio
        
        \bigskip
        \item Chiedere all'utente due numeri e stamparne la loro concatenazione
        \begin{itemize}
            \item \textit{Esempio: se l'utente inserisce 1 e 2, bisogna stampare 12}
        \end{itemize}
    \end{enumerate}
\end{exerciseframe}

\begin{exerciseframe}
    \frametitle{Esercizi (2)}
    \framesubtitle{Input}

    \begin{enumerate}
        \setcounter{enumi}{3}
        \item Chiedere all'utente il proprio nome e salutarlo. Il programma deve essere simile a questo:\\
            \texttt{[Python] Come ti chiami?}\\
            \texttt{[Utente] Davide}\\
            \texttt{[Python] Ciao Davide!}

        \bigskip
        \item Creare un programma che legga 3 numeri e stampi il minimo, il massimo e la media dei valori inseriti
        \begin{itemize}
            \item Per calcolare minimo e massimo potete usare le funzioni \fbox{\texttt{min}} e \fbox{\texttt{max}}
        \end{itemize}
    \end{enumerate}
\end{exerciseframe}


\begin{contentframe}
    \frametitle{Condizioni}

    \begin{columns}
        \col{.5}
        \begin{itemize}
            \item Permettono di eseguire delle azioni solo in certi casi
    
            \bigskip
            \item Sintassi:
            \begin{itemize}
                \item Dopo \texttt{if} c'è uno spazio
                \item La condizione ritorna un booleano (\texttt{True}/\texttt{False})
                \item Dopo la condizione ci sono i due punti (``\texttt{:}'')
                \item Le istruzioni eseguite all'interno della condizione sono indentate (con un tab o 4 spazi)
            \end{itemize}
        \end{itemize}
        
        \col{.5}
        \centering
        \image{if.jpg}
    \end{columns}
\end{contentframe}

\begin{contentframe}
    \frametitle{Condizioni}

    \begin{itemize}
        \item Se ho più condizioni, posso anche usare \texttt{else} o \texttt{elif} (\textit{else if})
    \end{itemize}

    \bigskip
    \centering
    \only<1>{
        \image{if_else.jpg}
    }
    \only<2>{
        \image[.5]{if_elif_bad.jpg}
    }
    \only<3>{
        \image[.5]{if_elif_good.jpg}
        
        \bigskip
        \texttt{elif} permette di mantenere il codice più facilmente leggibile
    }
    
\end{contentframe}

\begin{exerciseframe}
    \frametitle{Esercizi (1)}
    \framesubtitle{Condizioni}

    \begin{enumerate}
        \item Chiedere un numero e determinare se è pari

        \bigskip
        \item Chiedere all'utente un numero e determinare se è positivo, negativo o zero

        \bigskip
        \item Chiedere in input due parole e stampare quella con più caratteri
        \begin{itemize}
            \item Per vedere la lunghezza di una stringa, usare \boxed{\texttt{len(}\textit{stringa}\texttt{)}}
        \end{itemize}
    \end{enumerate}
\end{exerciseframe}


\begin{exerciseframe}
    \frametitle{Esercizi (2)}
    \framesubtitle{Condizioni}

    \begin{itemize}
        \item Chiedere all'utente in quale città abita
        \item Se la risposta è ``Genova'', rispondere con ``Anche io!''
        \item Altrimenti, rispondere con ``Io invece abito a Genova''

        \pause
        \bigskip
        \item Accettare come risposte corrette ``Genova'', ``genova'' e ``GENOVA''
        
        \pause
        \bigskip
        \item Opzionale: accettare tutte le possibili variazioni di case
    \end{itemize}
\end{exerciseframe}

\begin{exerciseframe}
    \frametitle{Esercizi di fine argomento}

    \begin{itemize}
        \item Chiedere all'utente un anno e dire se è bisestile
        \begin{itemize}
            \item \textit{Un anno è bisestile se è divisibile per 4...}
            \begin{itemize}
                \item \textit{ma non è divisible per 100}
                \item \textit{è divisible per 400}
            \end{itemize}
        \end{itemize}

        \bigskip
        \item Chiedere all'utente di inserire tre numeri, rispettivamente giorno, mese ed anno di una data. Stampare se la data inserita è una data valida o meno
    \end{itemize}
\end{exerciseframe}