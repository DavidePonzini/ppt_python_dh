\makesectionframe{Utilizzo del terminale}
\begin{contentframe}
    \frametitle{Il terminale}

    \begin{columns}
        \col{.5}
        \begin{itemize}
            \item I termini \textit{``terminale''}, \textit{``shell''} e \textit{``prompt dei comandi''} si riferiscono alla stessa cosa

            \bigskip
            \item Useremo il terminale spessissimo in questo corso, è molto importante imparare ad usarlo bene fin da subito
        \end{itemize}

        \col{.5}
        \centering
        \image[1][.7]{shell.jpg}
    \end{columns}
\end{contentframe}

\begin{contentframe}
    \frametitle{Utilizzo del terminale}

    \begin{itemize}
        \item Interfaccia grafica minimale per eseguire comandi e visualizzarne l'output
        \begin{itemize}
            \item Supporta solo formati testuali
        \end{itemize}

        \bigskip
        \item I comandi sono eseguiti quando premo \textbf{Invio}
        \item Eventuali parametri sono separati tramite il carattere \textbf{Spazio}
    \end{itemize}
\end{contentframe}

\begin{contentframe}
    \frametitle{Utilizzo del terminale}
    \frametitle{Cambiare cartelle}

    \begin{itemize}
        \item La riga corrente del prompt mi indica sempre in che cartella mi trovo
        \begin{itemize}
            \item Si può personalizzare per indicare anche altre informazioni, ma è fuori dallo scope di questo corso 
        \end{itemize}
        
        \bigskip
        \item Per cambiare cartella uso il comando \boxed{\texttt{cd}} \textit{(Change Directory)}
        \begin{itemize}
            \item \boxed{\texttt{cd NomeCartella}}
            
        \end{itemize}
    \end{itemize}
\end{contentframe}

\begin{exampleframe}
    \frametitle{Utilizzo del terminale}
    \frametitle{Cambiare cartelle -- errori comuni}

    \centering
    \boxed{\texttt{cd NomeCartella}}
    \bigskip
    
    \begin{itemize}
        \item Lo spazio è importante: separa il nome del comando dal nome del parametro
        \begin{itemize}
            \item \texttt{cdCartella} è un altro comando (che non esiste)
        \end{itemize}
        \bigskip
        
        \item Se la nostra cartella ha degli spazi bisogna racchiuderla tra virgolette
        \begin{itemize}
            \item \texttt{cd "La mia cartella"} oppure \texttt{cd \textquotesingle{}La mia cartella\textquotesingle{}} 
        \end{itemize}
        
        \bigskip
        \item I caratteri \texttt{"} e \texttt{\textquotesingle{}} sono diversi da ``, '', ` e '
        \begin{itemize}
            \item Particolarmente importante quando si fa copia-incolla
        \end{itemize}
    \end{itemize}
\end{exampleframe}

\begin{exampleframe}
    \frametitle{Utilizzo del terminale}
    \frametitle{Cambiare cartelle -- nomi speciali}

    
    \begin{itemize}
        \item Alcune cartelle hanno dei nomi speciali:
        \begin{itemize}
            \item \boxed{\texttt{.}}: cartella corrente
            \item \boxed{\texttt{..}}: livello superiore
            \item \boxed{\texttt{\tilde}}: livello superiore
        \end{itemize}
        \bigskip
        
        \item Se la nostra cartella ha degli spazi bisogna racchiuderla tra virgolette
        \begin{itemize}
            \item \texttt{cd "La mia cartella"} oppure \texttt{cd \textquotesingle{}La mia cartella\textquotesingle{}} 
        \end{itemize}
        
        \bigskip
        \item I caratteri \texttt{"} e \texttt{\textquotesingle{}} sono diversi da ``, '', ` e '
        \begin{itemize}
            \item Particolarmente importante quando si fa copia-incolla
        \end{itemize}
    \end{itemize}
\end{exampleframe}

\makesectionframe{Python}

\begin{exampleframe}
    \frametitle{Curiosità}

    \begin{itemize}
        \item Python non ha nulla a che fare con i rettili
        \item Prende il nome dallo show ``Monty Python's Flying Circus''
    \end{itemize}

    \bigskip
    \begin{columns}
        \col{.4}
        \centering
        \image{pythons1.jpg}
        
        \col{.4}
        \centering
        \image{pythons2.jpg}
    \end{columns}
\end{exampleframe}

\begin{contentframe}
    \frametitle{Python}

    \begin{itemize}
        \item Linguaggio di programmazione ad alto livello
        \begin{itemize}
            \item Sequenza di istruzioni che il computer esegue
            \item Si può fare qualunque cosa
            \item Alto livello: non mi devo preoccupare dei dettagli più fini
        \end{itemize}
        
        \bigskip
        \item Linguaggio interpretato
        \begin{itemize}
            \item Il codice non ha bisogno di essere compilato
        \end{itemize}

        \bigskip
        \item Esistono tanti moduli con funzionalità già fatte
        \begin{itemize}
            \item Permettono di risparmiare tempo
        \end{itemize}
    \end{itemize}
\end{contentframe}

\begin{exampleframe}
    \frametitle{Linguaggi a basso ed alto livello}

    \begin{itemize}
        \item Basso livello
        \begin{itemize}
            \item Devo indicare in dettaglio come eseguire qualsiasi azione
            \begin{itemize}
                \item \textit{Metti il valore 7 sul registro 3 della CPU, poi trasferisci il contenuto del registro 3 sulla cella 4728 della RAM} 
            \end{itemize}
            \item Esempio: Assembly
        \end{itemize}

        \bigskip
        \item Alto livello
        \begin{itemize}
            \item Molto più astratto, indico cosa fare e non come farlo
            \begin{itemize}
                \item \textit{\texttt{x = 7}}
            \end{itemize}
            \item Esempi: Python, C++, Java
        \end{itemize}
    \end{itemize}
\end{exampleframe}

\begin{exampleframe}
    \frametitle{Linguaggi compilati ed interpretati}

    \begin{itemize}
        \item Linguaggio compilato
        \begin{itemize}
            \item Tramite un programma apposito (compilatore), il codice viene convertito in istruzioni macchina
            \item Il programma risultante (.exe) si può poi eseguire direttamente senza più bisogno del compilatore
            \item Esempio: C, C++
        \end{itemize}

        \bigskip
        \item Linguaggio interpretato
        \begin{itemize}
            \item Le istruzioni sono lette ed eseguite da un programma apposito (interprete)
            \item Senza interprete, il codice non si può eseguire
            \item Esempi: Python, Java, Javascript
        \end{itemize}
    \end{itemize}
\end{exampleframe}

\begin{exerciseframe}
    \frametitle{Hello, world!}

    \begin{itemize}
        \item Eseguire il seguente programma: \fbox{\texttt{print(\textquotesingle{}Hello, world!\textquotesingle{})}}

        \bigskip
        \item Se lavoriamo in locale: creare un file con estensione .py, scrivere il codice, invocare l'interprete col seguente comando da terminale \fbox{\texttt{python nomedelfile.py}}
    \end{itemize}
\end{exerciseframe}

\makesectionframe{Concetti base}

\begin{contentframe}
    \frametitle{Funzioni}

    \begin{itemize}
        \item Sequenza di azioni che fa una cosa

        \bigskip
        \item Esempio: \fbox{\texttt{print(...)}} stampa dati su schermo

        \bigskip
        \item Sintassi: \texttt{nomedellafunzione(argomento1, argomento2, ...)}
        \begin{itemize}
            \item Le parentesi sono obbligatorie SEMPRE
            \item Gli argomenti sono i valori su cui voglio eseguire le operazioni
        \end{itemize}
    \end{itemize}
\end{contentframe}

\begin{exerciseframe}
    \frametitle{Funzioni}

    Cosa stampano le seguenti funzioni?
    \begin{itemize}
        \item \texttt{print(1)}\pause
        \item \texttt{print(5)}\pause
        \item \texttt{print(3, 5)}\pause
        \item \texttt{print(1, 2, 3, 4, 5)}\pause
        \item \texttt{print()}\pause
        \item \texttt{print}\pause{} -- Attenzione: non viene eseguita!
    \end{itemize}
\end{exerciseframe}

\begin{contentframe}
    \frametitle{Variabili}

    \begin{columns}
        \col{.5}
        \begin{itemize}
            \item Permettono di memorizzare dati o risultati

            \bigskip
            \item Possono avere (quasi) qualunque nome
            
            \bigskip
            \item Hanno diversi tipi, in base a che dati vogliamo salvare
            \begin{itemize}
                \item Python li capisce in automatico, ma è importante saperli riconoscere
            \end{itemize}
        \end{itemize}
        
        \col{.5}
        \centering
        \image{variables.jpg}
    \end{columns}
\end{contentframe}

\begin{contentframe}
    \frametitle{Variabili}
    \framesubtitle{Numeri}

    \begin{columns}
        \col{.5}
        \centering
        {\Huge\faCalculator}\\
        \bigskip
        \texttt{int}\\
        \textbf{Numero senza virgola}\\
        \bigskip
        1\\124\\-336

        \col{.5}
        \centering
        {\Huge\faCalculator}\\
        \bigskip
        \texttt{float}\\
        \textbf{Numero con virgola}\\
        \bigskip
        -3.14\\5.0\\\bigskip
    \end{columns}
\end{contentframe}

\begin{contentframe}
    \frametitle{Variabili}
    \framesubtitle{Testo}

    \begin{columns}
        \col{1}
        \centering
        {\Huge\faPen}\\
        \bigskip
        \texttt{str}\\
        \textbf{Stringa}\\
        \bigskip
        \texttt{"}ciao\texttt{"}\\
        \texttt{"}hello\texttt{"}\\
        \texttt{"}come stai?\texttt{"}
    \end{columns}
\end{contentframe}

\begin{contentframe}
    \frametitle{Variabili}
    \framesubtitle{Logica}

    \begin{columns}
        \col{1}
        \centering
        {\Huge\faCheck}\\
        \bigskip
        \texttt{bool}\\
        \textbf{Booleano}\\
        \bigskip
        True\\False
    \end{columns}
\end{contentframe}

\begin{contentframe}
    \frametitle{Variabili}
    \framesubtitle{Strutture dati base}

    \begin{columns}
        \col{.3}
        \centering
        {\Huge\faListOl}\\
        \bigskip
        \texttt{list}\\
        \textbf{Lista}\\
        (sequenza ordinata di elementi)\\
        \bigskip
        [1, 2, 2, 3]\\
        
        [\texttt{"}ciao\texttt{"}, \texttt{"}come\texttt{"}, \texttt{"}stai\texttt{"}]

        \col{.3}
        \centering
        {\Huge\faBars}\\
        \bigskip
        \texttt{set}\\
        \textbf{Set}\\
        (sequenza senza ripetizioni e senza ordine)\\
        \bigskip
        \{1, 5, 9\}\\
        \{True, False\}

        \col{.3}
        \centering
        {\Huge\faThList}\\
        \bigskip
        \texttt{dict}\\
        \textbf{Mappa/Dizionario}\\
        (coppie chiave-valore)\\
        \bigskip
        \bigskip
        \{\textquotesingle{}Paolo\textquotesingle{}: 8, \textquotesingle{}Alice\textquotesingle{}: 9\}
        \bigskip
    \end{columns}
\end{contentframe}

\begin{contentframe}
    \frametitle{Variabili}
    \framesubtitle{Altro}

    \begin{columns}
        \col{.5}
        \centering
        {\Huge\faSitemap}\\
        \bigskip
        \texttt{class}\\
        \textbf{Classi}\\
        \bigskip
        Insieme di dati di tipologie diverse

        \col{.5}
        \centering
        {\Huge\faTimes}\\
        \bigskip
        \texttt{None}\\
        \textbf{Nessun valore}\\
        \bigskip
        \bigskip
    \end{columns}
\end{contentframe}

\begin{exerciseframe}
    \frametitle{Variabili}

    Salvare in una variabile e stampare i seguenti valori
    \begin{itemize}
        \item 7\pause
        \item 5.5\pause
        \item ``Ciao come stai?''\pause
        \item True\pause
        \item{} [0, 1, 1, 2, 3, 5, 8, 13]\pause
        \item \{0, 1, 1, 2, 3, 5, 8, 13\}\pause
        \item I voti associati a tre studenti
    \end{itemize}
\end{exerciseframe}

\begin{contentframe}
    \frametitle{Operazioni matematiche}

    \centering
    \begin{tabular}{c|c|c|c}
        Operazione                      & Operatore     & Esempio           & Risultato         \\
        \midrule
        \midrule
        Somma                           & \texttt{+}    & \texttt{4 + 3}    & \texttt{7}        \\
        Sottrazione                     & \texttt{-}    & \texttt{4 - 3}    & \texttt{1}        \\
        Moltiplicazione                 & \texttt{*}    & \texttt{4 * 3}    & \texttt{12}       \\
        Divisione                       & \texttt{/}    & \texttt{4 / 3}    & \texttt{1.3333}   \\
        \midrule
        Elevamento a potenza            & \texttt{**}   & \texttt{4 ** 3}   & \texttt{64}       \\
        Divisione senza resto           & \texttt{//}   & \texttt{4 // 3}   & \texttt{1}        \\
        Modulo (resto della divisione)  & \texttt{\%}   & \texttt{4 \% 3}   & \texttt{1}        \\
    \end{tabular}
\end{contentframe}

\begin{contentframe}
    \frametitle{Operazioni logiche}

    \centering
    \begin{tabular}{c|c|c|c}
        Operazione                      & Operatore     & Esempio                   & Risultato         \\
        \midrule
        \midrule
        Maggiore                        & \texttt{>}    & \texttt{4 > 3}            & \texttt{True}     \\
        Maggiore o uguale               & \texttt{>=}   & \texttt{4 >= 3}           & \texttt{True}     \\
        Minore                          & \texttt{<}    & \texttt{4 < 3}            & \texttt{False}    \\
        Minore o uguale                 & \texttt{<=}   & \texttt{4 <= 3}           & \texttt{False}    \\
        Uguale                          & \texttt{==}   & \texttt{4 == 3}           & \texttt{False}    \\
        Diverso                         & \texttt{!=}   & \texttt{4 != 3}           & \texttt{True}     \\
        \midrule
        Negazione                       & \texttt{not}  & \texttt{not True}         & \texttt{False}    \\
        Congiunzione                    & \texttt{and}  & \texttt{True and False}   & \texttt{False}    \\
        Disgiunzione                    & \texttt{or}   & \texttt{True or False}    & \texttt{True}     \\
    \end{tabular}
\end{contentframe}

\begin{exerciseframe}
    \frametitle{Operazioni}

    Cosa stampa questo programma?

    \bigskip
    \image[.7]{exercise_logic.jpg}
\end{exerciseframe}

\begin{contentframe}
    \frametitle{Input}

    \begin{columns}
        \col{.5}
        \begin{itemize}
            \item La funzione \fbox{\texttt{input()}} permette di leggere un valore indicato dall'utente
    
            \bigskip
            \item Come argomento opzionale posso indicare la domanda che viene chiesta prima di leggere
        \end{itemize}
        
        \col{.5}
        \centering
        \image{input.jpg}
    \end{columns}

    \bigskip
    \begin{itemize}
        \item Attenzione: il valore ritornato è sempre di tipo \texttt{str}
        \item Per leggere un numero devo convertirlo in \texttt{int}
        \begin{itemize}
            \item \fbox{\texttt{x = int(input())}}
        \end{itemize}
    \end{itemize}
\end{contentframe}

\begin{exerciseframe}
    \frametitle{Input}

    \begin{itemize}
        \item Creare un programma che legga 3 numeri e stampi il minimo, il massimo e la media dei valori inseriti
        \begin{itemize}
            \item Per calcolare minimo e massimo potete usare le funzioni \fbox{\texttt{min}} e \fbox{\texttt{max}}
        \end{itemize}
    \end{itemize}
\end{exerciseframe}

