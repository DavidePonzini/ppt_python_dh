\begin{contentframe}
    \frametitle{File}

    \begin{columns}
        \col{.5}
        \begin{itemize}
            \item Due operazioni possibili
            \begin{itemize}
                \item Lettura
                \item Scrittura
            \end{itemize}
    
            \bigskip
            \item Processo per qualsiasi operazione:
            \begin{itemize}
                \item Aprire il file
                \item Fare le operazioni necessarie
                \item Chiudere il file
            \end{itemize}
        \end{itemize}
        
        \col{.5}
        \centering
        \image{file.jpg}
    \end{columns}
\end{contentframe}

\begin{contentframe}
    \frametitle{File}
    \framesubtitle{Keyword \texttt{with}}

    \begin{itemize}
        \item Con la keywork \texttt{with}, Python chiude il file in automatico
    \end{itemize}

    \bigskip
    \begin{columns}
        \col{.5}
        \image{file.jpg}
        
        \col{.5}
        \centering
        \image{file_with.jpg}
    \end{columns}
\end{contentframe}

\begin{contentframe}
    \frametitle{File}
    \framesubtitle{Modalità di apertura file}

    Il secondo argomento indica la modalità di apertura:
    
    \begin{columns}
        \col{.5}
        \begin{itemize}
            \item \texttt{r} \textit{(read)}
            \begin{itemize}
                \item Solo lettura (default)
            \end{itemize}
            \item \texttt{w} \textit{(write)}
            \begin{itemize}
                \item Solo scrittura
                \item Cancella il contenuto precedente
            \end{itemize}
        \end{itemize}
    
        \col{.5}
        \begin{itemize}
            \item \texttt{a} \textit{(append)}
            \begin{itemize}
                \item Solo scrittura
                \item Mantiene il contenuto precedente
            \end{itemize}
            \item \texttt{w+} \textit{(read-write)}
            \begin{itemize}
                \item Sia lettura che scrittura
            \end{itemize}
        \end{itemize}
    \end{columns}
\end{contentframe}

\begin{exerciseframe}
    \frametitle{File}

    \begin{itemize}
        \item Leggere il contenuto di un file e stampare la prima riga su schermo
        \bigskip
        \item Creare un file chiamato \texttt{prova.txt} che contenga ``Hello, world!''
        \bigskip
        \item Chiedere all'utente un numero e scriverlo in fondo al file \texttt{prova.txt}. Il contenuto precendente del file deve rimanere invariato
    \end{itemize}
\end{exerciseframe}



\begin{contentframe}
    \frametitle{Funzioni}

    \begin{columns}
        \col{.5}
        \begin{itemize}
            \item Blocchi di codice che fanno una cosa
            \item Posso riutilizzarli senza dover riscrivere tutto il codice ogni volta
            \item Permettono di organizzare meglio il codice
        \end{itemize}
        
        \col{.5}
        \centering
        \image{def.jpg}
    \end{columns}
\end{contentframe}

\begin{contentframe}
    \frametitle{Funzioni}
    \framesubtitle{Sintassi}

    \begin{columns}
        \col{.5}
        \begin{itemize}
            \item \texttt{def} indica che voglio definire una nuova funzione
            \bigskip
            \item \texttt{return} permette alla funzione di ritornare un risultato
            \begin{itemize}
                \item \texttt{return} interrompe l'esecuzione della funzione!
            \end{itemize}
            \bigskip
            \item Posso passare dei valori tramite argomenti
        \end{itemize}
        
        \col{.5}
        \centering
        \image{def.jpg}
    \end{columns}
\end{contentframe}

\begin{contentframe}
    \frametitle{Funzioni}
    \framesubtitle{Ricorsione}

    \begin{columns}
        \col{.5}
        \begin{itemize}
            \item Una funzione può a sua volta chiamare altre funzioni
            \bigskip
            \item Se una funzione chiama se stessa, si dice \textit{ricorsiva}
            \begin{itemize}
                \item Generalmente gli argomenti sono diversi
            \end{itemize}
        \end{itemize}
        
        \col{.5}
        \centering
        \image{def_fib.jpg}
    \end{columns}
\end{contentframe}

\begin{exerciseframe}
    \frametitle{Esercizi}

    \begin{enumerate}
        \item Creare una funzione che converte una temperatura Fahrenheit in Celsius
        \begin{itemize}
            \item $C = \frac{(F - 32) * 5}{9}$
        \end{itemize}
        \pause
        \bigskip
        \item Creare una funzione che ritorna \texttt{True} se un numero è pari, altrimenti \texttt{False}
        \pause
        \bigskip
        \item Creare una funzione che, con l'aiuto della funzione creata precedentemente, indica quanti numeri sono pari in una lista di numeri
        \pause
        \bigskip
        \item Creare una funzione che conta il numero di vocali in una stringa fornita dall'utente
    \end{enumerate}
\end{exerciseframe}

